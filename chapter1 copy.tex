\section{序論}\label{sec:intro}
\subsection{背景}\label{subsec:background}

\begin{comment}
近年インフラの老朽化が進み,インフラ維持管理の無人化および低コスト化が求められている.
日本では高度経済成長期に集中的に整備されたインフラが一斉に高齢化しており,国土交通省\cite{h25report}によれば,2033年までに建設後50年以上経過する
道路橋(橋長2m以上)の割合は,2018年時点の約25\%から約63\%まで増加する見込みとなっている.
インフラの中で日常的に管理されているのは比較的大規模なインフラだけであり,予算がないために日常的な実態調査ができていない例が多い.
地方公共団体の土木関係職員数は継続的に減少しており\cite{soumushou},インフラの維持管理に関する人員不足も懸念されている.
また,インフラの中には維持管理方法を導入する際に止められない例も多く,インフラを止めずに簡易的に導入できる維持管理方法が好ましい.
そのため,できるだけコストと人をかけずに簡単に導入できるインフラ維持管理方法が模索されている.
\end{comment}
高度成長期に建設された全国の土木構造物が一斉に耐用年数を迎え,その維持管理が重要な課題となっている.例えば,建設から50年を経過した道路橋(橋長2 m以上)の割合は,2018年時点で25 \%であったが,2033年には63 \%まで増加する見込みとなっている\cite{h25report}.こうした老朽化にも関わらず,十分に管理の行き届いている土木構造物は限定的である.その背景には財政的な問題に加え,行政における土木関係職員数の継続的な減少があり\cite{soumushou},それは人口減ないし税収減の著しい地方公共団体においてとりわけ顕著である.以上から,財政的および人的リソースをできるだけ必要とせず,容易に導入できる維持管理方法が模索されている.

\begin{comment}
インフラ維持管理の無人化・低コスト化を果たす上で,モニタリング技術の省力化は喫緊の課題である.
これまでインフラ構造物のモニタリングの多くは手作業による点検といった形で属人的に行われてきた.
しかし近年,ドローンや人工知能を活かした新たな手法に注目が集まっている.
ドローンによる人が行きにくい場所からの構造物の撮影や,人工知能による構造物の画像からのひび割れの自動検出等がその例である.
構造物を継続的に監視するモニタリング技術の発達は,適切かつ効率的なメンテナンスを行うことに大きく貢献しうる
\end{comment}
この点において,モニタリングにおける省力化は重要な課題である.従来の土木構造物のモニタリングは,手作業による点検など,労働集約的に行われてきた.しかし,近年出現したドローンや人工知能はそれぞれ,アクセスの困難な地点からの構造物の撮影や大量のデータを用いた構造物の異常検出を可能とするなど,新たな技術を生かしたモニタリング手法に注目が集まっている.モニタリング手法をうまく組み合わせることで,少ない人的・財政的コストであっても,適切かつ効率的な維持管理が可能となる.

\begin{comment}
動きのあるインフラ構造物のモニタリングにおいて,単眼カメラを用いたリモートセンシングは,人的・金銭的コストの小さい方法として有力である.
%動きのあるインフラ構造物モニタリングの省力化・低コスト化を図るうえで,単眼カメラにより得られた画像データを用いる方法は適している.
%無人で低コストなモニタリング手法として,単眼カメラによる画像データの取得は適している.
インフラ構造物のモニタリングデータの取得方法として,人の手作業による定期的な点検以外に,各種センサを用いた継続的なデータの取得が考えられる.
その中でも,対象物に触れることなく,離れたところから物体の形状や性質などを観測するリモートセンシングは,インフラを止めることなく導入可能なモニタリング方法の一つである.
単眼カメラは,ステレオカメラやLiDARといった他のリモートセンサと比べて,安価で普及している.
また,カメラは時間解像度,空間解像度が高い.
そのため,小さく速い動きのあるインフラ構造物に対してもその動きを正確にとらえることが可能である.
さらに,メンテナンスがしやすく,カメラの同期といった高度な処理も必要としないため,多くの人が簡単に利用可能である.
%一方で,画像のみからでは幾何学的に深度が定まらない,テクスチュアのない環境に弱いといった課題もある.
\end{comment}
数あるモニタリング手法の中でも,単眼カメラを用いたリモートセンシングは,諸コストがかからず汎用性が高い手法である一方,撮影対象を三次元的に扱うには工夫が必要である.対象物に触れることなく,離れたところから物体の形状や性質などを観測するリモートセンシングは,インフラの供用を止めることなく導入可能なモニタリング方法である.中でも,単眼カメラは安価で普及したリモートセンサであり,カメラから得られる画像は,ひび割れの記録や継続的な構造物の状況把握等,その用途は多岐にわたる.しかし,単眼カメラによって得られた二次元の画像は奥行き方向の距離(以下,深度)の情報を含んでいない.そのため,撮影対象を三次元的に扱うには,ステレオ撮影や,他のセンサから得られた深度情報と組み合わせる等の工夫が必要になる.これらの工夫によらず,単眼カメラのみからどれだけ三次元的な情報を復元できるかの限界を知ることは,構造物のモニタリングの可能性を考える上で大きな意味を持つ.

\begin{comment}
カメラの連続画像から静止している撮影対象物の三次元形状を逐次的に推定することは,「逐次的自己位置推定及び環境地図作成」(Simultaneous Localization and Mapping.以下,SLAM)というタスクにより達成できる.
これはセンサによって逐次的に周辺環境の地図を作成しながら,センサ自身の位置(自己位置)を推定する技術のことであり,自律移動するあらゆる機械の制御に欠かすことができない\cite{tomono正裕2018slam}.
SLAMは自動運転やドローンの自律飛行を目的として,近年活発に研究されている.
SLAMの中でも,カメラをセンサに用いる場合には特にVisual SLAMと呼ぶ.
これはその処理工程の中に,Structure from Motion(以下,SfM)と呼ばれる技術を含んでいる.
SfMはある対象を複数の視点から撮影した画像から,画像間の特徴点を対応づけることで三次元再構成と各視点姿勢を同時推定する技術である\cite{中村恭之2017-08-05}.
\end{comment}

ところで,移動するカメラの連続画像から撮影対象物の三次元形状を逐次的に推定することは,Visual Simultaneous Localization and Mapping(以下,Visual SLAM)というタスクにより達成できる.「逐次的自己位置推定及び環境地図作成」(Simultaneous Localization and Mapping.以下,SLAM)とは,センサによって逐次的に周辺環境の三次元地図を作成しながら,センサの自身の位置(以下,自己位置)を推定する技術のことであり\cite{tomono正裕2018slam},自動運転やドローンの自律飛行を目的として活発に研究されている.SLAMの中でも,カメラをセンサに用いる場合を特にVisual SLAMと呼ぶ(図\ref{fig:visual slam example}).元々Visual SLAMは撮影された周辺環境が静止しているという仮定の下で成り立つものであり,動きのある物体(以下,動的物体)を含む環境下では精度が急激に低下するという課題あった.そこで,動的物体と静止している部分(以下,静止背景)を分離し動的物体の影響を除くことで,近年精度の向上が図られている \cite{saputra2018visual}.

\begin{figure}
    \centering
    \includegraphics{images/3d_SLAM.png}
    \caption[Visual SLAMによる自己位置と三次元周辺環境地図推定の例]{Visual SLAMによる自己位置(青印)と三次元周辺環境地図推定の例(\cite{mur2017orb}より).}
    \label{fig:visual slam example}
\end{figure}

\begin{comment}
動的物体を含む環境下でのVisual SLAM(以下,動的Visual SLAM)に対しては,動的物体と静的な背景を個別に扱うことで精度を向上させる技術が近年発達した\cite{saputra2018visual}.
ここ数十年でVisual SLAMは急激な発展を遂げてきたが,ほとんどの技術は外的環境が変化しないという想定の下で成り立っているものであった.
そのため,動物体を含む環境に対しては,特徴点の誤対応やオクルージョン(注目している物体が前にある物体に遮られること)により精度が急激に落ちるという問題点があった\cite{tan2013robust}.
これを克服したのが,動的物体と静的な背景を個別に扱う方法である\cite{saputra2018visual}.
具体的には以下の手順を踏む.
\begin{enumerate}
 \item 入力された連続画像を動的物体と静的な背景に分離する.
 \item 画像における静的な背景の部分のみから特徴点を抽出し,それらを用いて従来のSLAMを行うことで,カメラ姿勢推定と静的な背景のみの環境地図の構築を行う.
 \item 上の手順で得られたカメラ姿勢と分離された動的物体の深度推定などから,動的物体の三次元復元および追跡を行う.
\end{enumerate}
しかし,動的Visual SLAMにおける動的物体自体の形状や変位の把握は依然としてチャレンジングな課題である.
動的Visual SLAMに関する研究の多くは,カメラの自己位置推定および静的な背景の復元に重きを置いてきた.
動的物体の部分については,動的物体が直線や円錐曲線上を動くと仮定して追跡する\cite{alcantarilla2012combining,avidan1999trajectory}など,動きに対して何かしらの仮定が設けられていることが多い.
深層学習ベースの単眼深度推定を用いた動的物体の形状復元も手法として考えられるが,こちらは学習していない物体に対しては十分な精度が得られないという問題がある.
そのため,動的物体の複雑な動きを捉えることや,高精度な位置推定・微小な変位推定を行おうとしたVisual SLAMの研究は少ない.
\end{comment}

しかし,動的物体を含んだVisual SLAM(以下,動的Visual SLAM)において,動的物体自体の形状や変位の把握は依然としてチャレンジングな課題である.従来の動的Visual SLAMに関する研究の多くは,カメラの自己位置推定および静止背景の三次元地図推定の精度向上に重きを置いてきた.静止背景から分離された動的物体の部分については,動的物体が直線や円錐曲線上を動くと仮定して追跡する\cite{alcantarilla2012combining,avidan1999trajectory}など,動きに対して何らかの仮定が設けられていることが多い.そのため,動的物体の複雑な動きを推定できる動的Visual SLAMの手法が模索されている.

そこで本研究では,従来まで画像から動的物体を除いた静止背景にのみに適用していたVisual SLAMの手法を,あえて動的物体に対して適用することで,その形状と動きを復元する手法を提案する.
%セグメンテーション技術を用いて動的物体の領域と静止背景の領域を分離し,注目したい動的物体の領域のみに対してVisual SLAMを行う.
%Visual SLAMの出力として推定された,動的物体の三次元形状と動的物体に対するカメラの相対的な自己位置を用いて,動的物体の各時刻における三次元位置の復元が可能になる.
土木構造物のモニタリングにおいて,単眼カメラのみからでも動きのある構造物の三次元形状や各部分の変位を捉えることができれば,そこから画像上のひび割れの三次元位置の把握や,注目したい部分の画像間の対応関係の把握などを行うことができるようになると期待される

\begin{comment}
そこで,画像における動的な部分の特徴点のみを用いたSLAMを導入することで,動的物体の形状を復元することを可能にする.
従来の方法では静的な背景の特徴点に対してのみSLAMを適用していた.これによって,静的な背景の環境地図とカメラの自己位置を求めることができる,しかし,それとは別に動的物体上の特徴点のみでSLAMを行うことで,動的物体に対するカメラの相対位置と動的物体の三次元形状を同時推定することができる.
そしてカメラの自己位置と動的物体に対するカメラの相対位置を重ね合わせることで,周辺環境地図内での動的物体の三次元位置の復元が可能になる.

動的物体の三次元位置や変位を高い精度で推定するためには,動的物体と静的な背景を正確に分離することと画像上の点の移動を正確に捉えることが必要になる.
近年は機械学習の著しい進歩により,コンピュータービジョンの分野においてセグメンテーション,およびオプティカルフロー推定と呼ばれる技術の研究が進展している.
セグメンテーションとは画像中に含まれる物体を認識し,それが存在する領域を検出する技術である.近年は特に深層学習に基づくセグメンテーション手法が,未知の物体に対しても高い精度を示すようになった\cite{cheng2022xmem}.
セグメンテーションを用いて動的物体と静的な背景の特徴点を正確に分離することで,SLAMを適用する際に動的な特徴点と静的な特徴点が混在することを防げる.
オプティカルフロー推定とは,連続する2フレーム間で対応する画素や特徴量のフロー(画像座標上での変位)を推定する技術である\cite{horn1981determining}.
従来オプティカルフローは,人の手によって作られた最適化目標を解く問題として推定されてきた\cite{chen2016full,horn1981determining,zach2007duality}が,近年はフローを直接推定する深層学習に基づく手法が,従来の手法を精度・速度の面で上回っている\cite{sun2018pwc,yang2019volumetric,ilg2017flownet}.
オプティカルフロー推定の精度を向上させることで,物体の各画素の時空間的な動きを正確に捉えることができる.

インフラのモニタリングにおいて,動的構造物の形状や各部分(画素)の変位を高精度でとらえることができれば,そこから画像上のひび割れの三次元位置の把握や,注目したい部分の画像間の対応関係の把握などを正確に行うことができるようになると期待される.
\end{comment}


\begin{comment}
動的物体を含む環境下でのVisual SLAMに関する研究の多くは,車といった動的物体を剛体と仮定し,カメラの姿勢推定および静的な背景の復元に重きを置いてきた.
そのため,動的物体の形状変化や表面の微小な変位をとらえようとしたVisual SLAMの研究は少ない.
動的な前景を物体ごとにセグメンテーションする際や動的物体を追跡する際に,物体を剛体仮定している既存手法が多いためである.
そのほかにも,動的物体が直線や円錐曲線上を動くと仮定して追跡する\cite{alcantarilla2012combining,avidan1999trajectory}など,動的物体の複雑な動きをとらえることのできる手法が適用された例は少ない.
動的物体の形状変化や微小な変位を高精度でとらえることができれば,そこから物体のひずみなどの有用な情報を得ることができる.

近年は機械学習の著しい進歩により,コンピュータービジョンの分野において深層学習に基づいた様々な技術の精度が向上している.
単眼深度推定やオプティカルフロー推定と呼ばれる技術の精度向上はよい例である.
単眼深度推定とは,単眼カメラでとらえた映像からそこに写っているものとの距離を推定する技術である.
単眼カメラからの画像のみでは,幾何学的に深度情報が得られない.
そこで近年,深層学習を用いて単眼カメラの画像のみから深度推定を行う手法が盛んに研究されてきた.
オプティカルフロー推定とは,連続する2フレーム間で対応する画素や特徴量のフロー(画像座標上での変位)を推定する技術である\cite{horn1981determining}.
従来オプティカルフローは,人の手によって作られた最適化目標を解く問題として推定されてきた\cite{chen2016full,horn1981determining,zach2007duality}が,近年はフローを直接推定する深層学習に基づく手法が,従来の手法を精度・速度の面で上回っている\cite{sun2018pwc,yang2019volumetric,ilg2017flownet}.
Visual SLAMにおいて深度推定の精度を向上させることで,動的物体の形状復元を高精度で行える.
また,オプティカルフロー推定の精度を向上させることで,物体の各画素の時空間的な動きを正確に捉えることができる.

動的物体と静的な背景のセグメンテーションを行ったうえで,深度推定とオプティカルフロー推定を動的物体に対して高精度に行うことで,動的物体の形状変化や微小な変位の追跡を行うことが可能となる.

\end{comment}


\subsection{目的}\label{subsec:objective}

本研究では,「単眼カメラを用いた動的構造物の三次元復元」を目的とする.
具体的には,剛体運動をする動的構造物の三次元復元手法の構築.
そして,動的構造物の形状推定・三次元空間上での追跡の高精度化である.
%そして,動的構造物の各画素における変位を捉え,三次元空間上で追跡することである.

\subsection{本論文の構成}\label{subsec:construction}

本論文は,本章を含めて全5章で構成される.

本章では,本研究の背景・目的を述べた.

第2章では,既往研究の整理を行う.

第3章では,本研究の提案手法を紹介する.

第4章では,提案手法を適用した結果とその精度について議論する.

第5章では,本論文の結論を示した後,今後の発展可能性を示す.