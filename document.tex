\documentclass{jsarticle} % pLaTeX を使う場合
%\documentclass[uplatex]{jsarticle} % upLaTeX を使う場合。しおりが文字化けする
\bibliographystyle{junsrt}%番号順
\usepackage{atbegshi}
\AtBeginShipoutFirst{\special{pdf:tounicode 90ms-RKSJ-UCS2}}%文字コード
\usepackage{amsmath}%数式
\usepackage{amssymb}%数式(追加)
\usepackage{amsfonts}%数式(追加)
\usepackage{tabularx}%幅が可変な表など
\usepackage{float}%図表の出力位置として[H]を使えるようにする
\usepackage{multirow}%行の結合
\usepackage{longtable}%ページ跨ぎの表
\usepackage[dvipdfmx]{graphicx, xcolor}%図・色を使う
\usepackage{bm}%ベクトル用太斜体
\usepackage[normalem]{ulem}
\usepackage{lscape}%横向き
\usepackage{fancyhdr}%ヘッダ
\usepackage{listings}%ソースコードリストjlistingを追加しよう
\usepackage{comment}%複数行コメント
%\usepackage{geometry}%ページごと余白
\usepackage{url}
\usepackage[hang,small,bf]{caption}
\usepackage[subrefformat=parens]{subcaption}
\usepackage{booktabs}
\captionsetup{compatibility=false}
\usepackage[dvipdfmx,%
bookmarks=true,%しおりを作るか
bookmarksnumbered=true,%しおりに節番号をつけるか
bookmarksopen=true,%しおりのツリーを開くか
anchorcolor=black,%アンカーテキストの色指定
citecolor=green,%参考文献リンクの色指定
filecolor=magenta,%ローカルファイルリンクの色指定
linkcolor=red,%pdfファイルのリンクの色指定
urlcolor=magenta,%外部参照しているURLの色指定
colorlinks=false,%リンクの文字に色をつけるか
linkbordercolor={0 0 0},%リンクを囲むボックスの色
pdfborder={0 0 0},%
pdftitle={幾何学的条件を考慮した Defocusing 3D Gaussian Splatting},%
pdfsubject={2024年度東京大学大学院工学系研究科社会基盤学専攻修士論文},%
pdfauthor={小林 健},%
pdfkeywords={},%
]{hyperref}%しおり・文書情報

%----------目次の設定----------
\setcounter{tocdepth}{3}%目次でどこまで出力するか

%----------論文要旨のレイアウト----------
\renewcommand{\abstractname}{論文要旨}
\renewenvironment{abstract}{%
	\setlength{\headsep}{0pt}
	\setlength{\headheight}{0pt}
	\setlength{\textheight}{39\baselineskip}
	\addtolength{\textheight}{\topskip}
	\begin{center}{\fontsize{12pt}{20pt}\selectfont \textgt \abstractname}\end{center}
	\fontsize{10.5pt}{19pt}\selectfont%文字サイズ,行間
}%要旨の余白・フォント設定

%----------ソースコードの設定----------
\renewcommand{\lstlistingname}{リスト}%ソースコードのコメントに日本語がある場合,「//」の他,全角カンマや半角引用符の後ろに半角スペースを入れないと出力がおかしくなるバグあり
\lstset{%ソースコードの表示設定
	language={C++},%
	backgroundcolor={\color[gray]{.95}},%
	basicstyle={\small\ttfamily},%標準書体
	identifierstyle={\small},%
	commentstyle={\small\rmfamily \color[rgb]{0,0.5,0}},%コメント書体
	classoffset=0,%関数名等の色の設定
	keywordstyle={\small\sffamily\bfseries \color[rgb]{1,0,0}},%intやifの書体
	ndkeywordstyle={\small},%
	stringstyle={\small\ttfamily \color[rgb]{0,0,1}},%表示する文字の書体
	frame={tb},%
	breaklines=true,%自動改行
	breakindent=10pt,%改行後のインデント量(デフォは20pt)
	columns=[l]{fullflexible},%
	numbers=left,%行番号の位置
	tabsize=4,%タブの大きさ
	xrightmargin=0zw,%
	xleftmargin=3zw,%
	numberstyle={\scriptsize},%行番号の書体
	stepnumber=1,%行番号の間隔
	numbersep=1zw,%
	lineskip=-0.5ex%
}

%----------謝辞のレイアウト----------
\newenvironment{heightchange}{%
	\setlength{\textheight}{680pt}
}%謝辞用の本文領域の高さ設定(ページが変わる前じゃないと効かない)

\makeatletter

%-----enumerate環境の設定-----
\renewcommand{\p@enumii}{}%番号付箇条書きenumerate環境における2段階目のlabelとrefの出力を変更

%-----数式番号を(章-節毎の番号)に変更-----
\renewcommand{\theequation}{%
\thesection.\arabic{equation}}
\@addtoreset{equation}{section}

%-----図番号を(章-節毎の番号)に変更-----
\renewcommand{\thefigure}{%
\thesection.\arabic{figure}}
\@addtoreset{figure}{section}

%-----表番号を(章-節毎の番号)に変更-----
\renewcommand{\thetable}{%
\thesection.\arabic{table}}
\@addtoreset{table}{section}

%-----longtableにおけるcaptionの付け方をjsarticleと同様に-----
\def\LT@makecaption#1#2#3{%
	\LT@mcol\LT@cols c{\hbox to\z@%
		{\hss\parbox[t]\LTcapwidth{%
				\sbox\@tempboxa{#1{\normalsize #2\hskip1zw\relax}\normalsize#3}%
				\ifdim\wd\@tempboxa>\hsize
				#1{\normalsize #2\hskip1zw\relax}\normalsize #3%
				\else%
				\hbox to\hsize{\hfil\box\@tempboxa\hfil}%
				\fi%
				\endgraf\vskip\baselineskip}%
			\hss}}}%

%----------表紙のレイアウト----------
\def\@thesis{2024年度修士論文}
\def\@ChiefExaminer{主査:}
\def\@SubChiefExaminer{副査:}
\def\id#1{\def\@id{#1}}
\def\EngTitle#1{\def\@EngTitle{#1}}
\def\department#1{\def\@department{#1}}
\def\ChiefName#1{\def\@ChiefName{#1}}
\def\SubChiefName#1{\def\@SubChiefName{#1}}

\def\@maketitle{
	\begin{center}
		{\Large \@thesis \par}
		\vspace{10mm}
		{\huge \bf \@title \par}
		\vspace{5mm}
		{\Large \@EngTitle \par}
		\vspace{10mm}
		{\Large \@id \@author \par}
		\vspace{40mm}
		{\Large \bf \@ChiefExaminer}
		{\Large \@ChiefName \par}
		\vspace{3mm}
		{\Large \bf \@SubChiefExaminer}
		{\Large \@SubChiefName \par}
		\vspace{10mm}
		\begin{tabularx}{100mm}{|X|r|r|}
			\hline
			\multicolumn{1}{|c|}{署名} & \multicolumn{1}{c|}{ 日付 } & \multicolumn{1}{c|}{ 印 } \\ \hline
			\rule[-2mm]{0mm}{14mm} 
			\begin{minipage}{50mm}
			\end{minipage}%
			&%
			\begin{minipage}{15mm}
			\end{minipage}%
			&%
			\begin{minipage}{15mm}
			\end{minipage}%
			\\ \hline
			\rule[-2mm]{0mm}{14mm} 
			\begin{minipage}{50mm}
			\end{minipage}%
			&%
			\begin{minipage}{15mm}
			\end{minipage}%
			&%
			\begin{minipage}{15mm}
			\end{minipage}%
			\\ \hline
		\end{tabularx}
		\par
		\vspace{20mm}
		{\Large \@department \par}
		\vspace{10mm}
		{ \@date }
		\vspace{10mm}
	\end{center}
	\par\vskip 1.5em
}

\makeatother

%----------表紙の内容をここに書く----------
\title{幾何学的条件を考慮した\\Defocusing 3D Gaussian Splatting}
\date{2025年1月21日}
\EngTitle{Defocusing 3D Gaussian Splatting\\Considering Geometric Condition}
\department{東京大学大学院 工学系研究科 社会基盤学専攻}
\id{37-236014}
\author{小林 健}
\ChiefName{布施 孝志 教授}
\SubChiefName{竹内 渉 教授}

\begin{document}
\pagenumbering{Roman}%ページ番号を大文字のローマ数字にする
%----------表紙----------
\maketitle
\thispagestyle{empty}%ヘッダ・フッター・ページ番号なし
\newpage

%----------要旨----------
\thispagestyle{empty}%ヘッダ・フッター・ページ番号なし
\input abstract.tex
\newpage

%----------目次・図目次・表目次のページスタイルの設定----------
\pagestyle{fancy}
\rhead{}%ヘッダ右はなし
\lhead{\rightmark}%ヘッダ左は[目次,図目次or表目次]と表示
\pagenumbering{roman}%ページ番号を小文字のローマ数字にする

%----------目次----------
\tableofcontents
\newpage

%----------図目次----------
\listoffigures
\newpage

%----------表目次----------
\listoftables
\newpage

%----------本文のページスタイルの設定----------
\pagestyle{fancy}
\rhead{\rightmark}%ヘッダ右に章見出し
\lhead{\leftmark}%ヘッダ左に節見出し
\pagenumbering{arabic}%ページ番号をアラビア数字にする

% %----------第1章---------
% \section{序論}\label{sec:intro}
\subsection{背景}\label{subsec:background}

インフラ構造物の施工から維持管理の段階では大量の画像が撮影されており,現況の直感的な把握が可能な画像管理方法が模索されている.
構造物の状況を記録・把握するうえで,画像の使用は低コストかつ汎用性が高い方法である.
特に,近年ドローン等の普及に伴い,より多くの視点から画像が取得可能になった. 
しかし,このようにして得られた大量の画像は,必要な情報に対応する画像の参照や,撮影場所の正確な特定,画像間の相互参照に適したデータベースとしては集約されておらず,直感的な現況の把握は困難である.
例えば,施工現場におけるの記録画像の管理は電子小黒板と紐づけた階層的な管理がされている程度\cite{kouji-shashin}である.
以上から,現況の撮影画像をより現実に近い形で参照できる方法が求められている.
この点において,大量の画像群を三次元空間上で集約・統合するプラットフォームの構築は,現況の直感的な把握に大きく寄与する.
撮影画像と三次元空間が紐づくことで,画像と撮影位置の対応が明確になり,迅速な位置特定や近隣画像の相互参照が可能になる.
また,大量の画像情報を一つの三次元空間表現に集約し,未撮影視点での画像取得が可能になれば,仮想空間上で現実に近い形での現況把握や分析が実現できる.\par

しかしながら,画像を用いた三次元復元に関する一般的な手法は,現実に即した見た目の再現に課題がある.
具体的には,(1)軽量な三次元モデルで複雑な形状を再現できない,(2)視点によって見え方が異なる部分の復元に失敗する,という二つの課題がある.
ここでいう三次元モデルとは,点群やポリゴン,後述するガウシアン等により,三次元空間を表現するモデルを指す.
撮影画像群からの三次元復元では,Structure from Motion(以下,SfM)とMulti View Stereo(以下,MVS)という二つの手法を組み合わせることが一般的である.
撮影画像群を入力として,SfMで疎な三次元点群とカメラ標定要素を推定し,MVSで密な三次元モデルを作成する.
しかし,一般的なMVSは,ポリゴンや点群など離散的な空間表現を用いるため,細部の複雑な形状を表現するには高い計算コストとデータ容量を要する.
また,反射や透過等の原因で視線方向により見え方が異なる部分では,SfMの段階で特徴点同士の対応付けが困難となり,復元が失敗する場合がある.\par

一方で,複雑な形状や視点による見え方の違いを再現できる,フォトリアルな三次元復元手法として,Differentiable Renderingという手法が近年注目を集めている.
Differentiable Renderingとは,レンダリング画像と入力画像の差分に基づき,三次元モデルを調整できるレンダリング手法の総称である\cite{differentiable_rendering}.
レンダリング画像と入力画像を直接比較し三次元モデルのパラメタを最適化するため,現実に近いレンダリング画像が再現できる.
また,三次元モデル作成時に入力画像間の対応関係を必要としないため,視線方向による見え方の違いに対しロバストである.
Differentiable Renderingの中でも,2023年に発表された3D Gaussian Splatting(以下,3DGS)\cite{3dgs}は,視覚的再現性の高さとレンダリングの速さから,現在コンピュータービジョンの分野で盛んに研究されている.
3DGSとは,色を持った三次元の異方性ガウシアンの集合により空間を表現する手法である.
各ガウシアンは,三次元位置に加え,共分散行列,不透明度,色を構成要素に持つ.
異方性を持つガウシアンを用いることで,ポリゴンや点群と比べて少ないパラメタで複雑かつ連続的な表現を実現している(図:\ref{fig:green-house}).
また,ガウシアンの持つ色を視線方向の関数とすることで,視線方向による見え方の違いを表現する.\par

\begin{figure}[h]
    \centering
    \includegraphics[width=0.5\linewidth]{images/chapter1/3dgs_green_house.png}
    \caption[3DGSによる三次元空間表現の例]{3DGSによる三次元空間表現の例(\cite{green-house}より).植物の葉のような細部の複雑な形状表現や,水面の反射を再現している.}
    \label{fig:green-house}
\end{figure}

しかし,入力画像にピンボケや手ブレといった不鮮明な画像が含まれている場合の三次元復元は,3DGSであっても依然としてチャレンジングな課題である.
その原因の一つは,ガウシアンが入力画像のノイズに過剰に適合するからである.
不鮮明な部分を再現するようにガウシアンが学習され,レンダリング時にピンボケやブレを含んだ画像が生成される.
ピンボケへの対応を試みている3DGS手法\cite{Deblurring3dgs}\cite{BAGS}もあるが,これらの手法では入力画像を再現するピンボケの表現が深層学習ベースで学習されている
そのため,ガウシアンがピンボケに過剰に適合し,カメラからの深度に応じてピンボケの大きさが変化するという幾何学的条件が満たされず失敗する場合がある.
上記の要因に加え,3DGSの入力となるカメラ標定要素が不正確になることも,正確な三次元復元ができない一因である.
3DGSではSfMで推定されたカメラ標定要素を入力として用いる.
そのため,特徴点の抽出が困難である不鮮明な画像では,SfMによる推定カメラ標定要素が不正確となり,3DGSによる三次元モデルの構成要素に大きな誤差が生じる場合がある.
しかし,ピンボケ画像の入力に適用可能な既存手法はカメラ標定要素が正しい場合を前提としており,カメラ標定要素が不正確な場合に修正する手法は管見の限りではない.\par

そこで,本研究では, 幾何学的条件に基づくピンボケ表現を導入するとともに,不正確なカメラ標定要素の修正を行う3DGS手法を提案する.
ピンボケ画像を含んだ画像群からロバストにフォトリアルな三次元空間再現が可能になることで,撮影された画像情報を無駄なく利用でき,より効率的な現況把握が可能になると期待される.\par

\subsection{目的}\label{subsec:objective}

本研究では,「ピンボケ画像群に対しロバストな3D Gaussian Splatting手法の構築」を目的とする.
具体的には,カメラの原理に基づくピンボケ表現の導入と不正確なカメラ姿勢の修正を伴う三次元モデルの最適化を行う.
ピンボケの大きさはカメラから被写体までの深度に依存し一意に定まる.
そこで本研究では,ガウシアンの深度に基づく正則化項を損失に追加することで,幾何学的条件に忠実なピンボケ表現を実現する.
また,SfMの出力するカメラ標定要素が不正確な場合を想定し,カメラ標定要素の修正とピンボケ表現を伴うガウシアンの更新を逐次的に行う3DGS手法を構築する.
本研究では上記手法の有効性を検証するため,既存のデータセットに加え,カメラ標定要素が評価可能なデータセットを作成し,評価を行う.
既存のデータセットでは,ピンボケ画像を用いてSfMを行った結果がカメラ標定要素として使用されるため,カメラ標定要素に誤差が含まれている可能性がある.
そこで,本研究では複数の焦点距離画像が取得可能なカメラで撮影を行い,同一のカメラ標定要素を持つピンボケ画像と全焦点画像を用いてデータセットを作成する.
全焦点画像に得られるカメラ標定要素を真値とすることで,カメラ標定要素修正の有効性を評価する.

\subsection{本論文の構成}\label{subsec:construction}

本論文は,本章を含めて全5章で構成される.

本章では,本研究の背景・目的を述べた.

第2章では,既往研究の整理を行う.

第3章では,本研究の提案手法を紹介する.

第4章では,提案手法を適用した結果とその精度について議論する.

第5章では,本論文の結論を示した後,今後の発展可能性を示す.
% \clearpage

% %----------第2章----------
% \section{既往研究の整理}\label{sec:researches}
本章では,本研究に関係する既往研究を整理する.\par
\subsection{Structure from MotionとMulti View Stereo}\label{subsec:Sfm_and_MVS}
ここでは画像群のみからの環境の三次元復元において,現在の主要な手法であるStructure from Motion(以下,SfM)とMulti View Stereo(以下,MVS)について述べる.
この手法は異なる視点から撮影された画像群を入力として,まずはSfMを用いて物体の疎な三次元点群と各画像の三次元位置と姿勢を出力する.
SfMで得られた出力と画像を入力とし,MVSで点群を密にしていく.\par

SfMは,移動するカメラから得られる画像から,画像に写った対象物の幾何学的形状とカメラの動きを同時に復元する手法である\cite{sfm_JP}.
SfMは特徴点抽出・マッチングとバンドル調整という二つの段階に大別される.特徴点抽出ではSIFT\cite{SIFT}やORB\cite{ORB}といった特徴量記述子を用いて,エッジやコーナーといった特徴的な画像状の点を抽出する.
近年は深層学習を用いた特徴点記述子の研究も多く進められている.
こうして得られた特徴点を,対応する特徴量を用いて画像間で対応させていくことで,共通した三次元点を映し出す画像座標の対応を得ることができる.この際,RANSAC(Random Sample Consensus)\cite{RANSAC}を適用し,拘束条件(共線条件もしくはエピポーラ拘束)を満たさない誤対応点を排除する等の措置が取られる.
画像座標の対応が得られたら,カメラの位置・姿勢と対応点の三次元座標をバンドル調整を用いて計算する.
バンドル調整では,対応点の三次元座標を再び画像に投影した際の画像座標の誤差(再投影誤差)が最小になるように,非線形最小二乗法によってカメラの位置・姿勢と対応点の三次元座標の同時最適化が行われる.\par

MVSでは,三枚以上の画像を同時利用して画像間の対応に基づく密な三次元点群の生成を行う.
MVS手法の多くはマッチングコスト(類似度)の計算,周辺領域におけるコストの集約,視差の計算と最適化,視差の改善という4つのステップで行われ\cite{mvs},計算された視差に基づいて画像上の各点に対応する三次元点群が密に生成される.\par

以上のように,SfMやMVSは画像間の対応関係に基づいて三次元復元を行っているため,特徴点が取りづらい場合や特徴点間の対応関係がつけづらい場合に失敗するという課題がある.
前者については,ピンボケやカメラの手ブレ,テクスチャの少ない被写体(白い壁が続く廊下など)といった状況があげられる.
後者については,鏡面反射や透過といった見る方向によって見え方が異なる場合,物体が他の物体に隠れてしまう場合(以下,オクルージョン)などがあげられる.
このような状況では,SfMによるカメラ位置・姿勢の推定に大きな誤差が含まれたり,最悪の場合姿勢推定が不可能になる.
また,特徴点間の対応関係がつけづらい部分が復元されないということも起こる.\par

\subsection{Differentiable Rendering}\label{subsec:Differentiable Rendering}
画像群を入力として,撮影されていない新たな視点からの画像レンダリング(以下,Novel View Synthesis)という文脈で,Differentiable Renderingという手法が近年活発に研究されている.\par

1990年代以前,Novel View Synthesisでは三次元モデルの作成とモデルからの画像のレンダリングは分離されていた.
そのため,モデル作成時に画像情報が欠損するという理由から,視覚的再現性が下がるという課題があった.
そこで,レンダリング時に,(三次元モデルに加え)画像情報を直接使用するImage-based Renderingという手法群が研究されてきた.
Image-based Renderingは明示的な三次元モデルをどれだけ利用するかに応じて大別される\cite{image_based_rendering}.
明示的な三次元モデルを用いないPlenoptic Modeling\cite{pleptonic_modeling}やLight Field\cite{light_field},特徴点の対応関係といった画像間の位置対応を利用するview interpoltion\cite{view_interpolation},画像の深度情報を用いて明示的に三次元情報を表現する3D Warping\cite{3d_warping}といった手法が主である.\par

三次元モデルの勾配を,画像のみの比較を通じてend-to-endで計算,伝播できるようにImage-based Renderingが拡張されたレンダリング手法群がDifferentiable Renderingである.
Differentiable Renderingではレンダリングした画像と撮影画像を直接比較するため,画像間の対応関係を用いる必要がなく,鏡面反射や透過といった見る方向によって見え方が異なる場合に対しても頑強である.
Differentiable Renderingは,Image-based Renderingの強みである視覚的再現性の高さと,幾何学的な条件を考慮できるという三次元モデルを使用した手法の強みを持ち合わせている一方で,レンダリングシステムが複雑であることやモデル構築に多くの写真が必要であるという欠点もある\cite{differentiable_rendering}.
Differentiable Renderingは三次元の表現方法に応じて,メッシュベース,ボクセルベース,点群ベース,陰関数ベースの手法に大別される.\par

中でも,陰関数ベースの手法であるNeural Radiance Field(以下,Nerf)\cite{nerf}は,その視覚的再現性の高さからコンピュータービジョン分野で多くの研究がなされている.
Nerfは三次元空間をradiance fieldと呼ばれる場で表現し,radiance fieldをニューラルネットワークで学習・表現した手法である.
このネットワークは,三次元位置$(x,y,z)$と視線方向$(\theta, \phi)$を入力として,色$c$と密度$\sigma$を出力する.
レンダリングをする際は,光線上の点をサンプリングし,密度$\sigma$を重みとして色を積分していくことで画素値が定まる.
色が視線方向の関数となることで反射といった視線方向による見え方の違いを表現できるとともに,密度に基づいて色を積分するため光の透過も表現できる.
Nerfは陰関数ベースの手法であるため,連続的な表現が可能であり,少ないメモリで複雑な表現が可能という強みがある.
一方,光線上の点のサンプリングは計算コストが高いことに加え,空間全体を学習する必要があるため,学習やレンダリング時に多くの時間がかかる(学習は数時間,レンダリングは0.1FPS以下)という問題点がある\cite{3dgs}.
ボクセルやハッシュグリッドを用いて離散的にradiance fieldを表現することで学習とレンダリングを高速化する手法もある\cite{plenoxels}\cite{instant-ngp}ものの,レンダリング速度は10FPS程と,リアルタイムレンダリングは難しい.


% \subsection{用語および記号の定義}\label{subsec:definition}

% ここでは本論文で用いる用語と記号を定義する.

% まず,本論文における自己位置とは,センサの3次元座標$\mathbf{t}={
% 	(t_{(x)} , 
% 	t_{(y)} ,
% 	t_{(z)})
% }^\mathrm{T}\in\mathbb{R}^3$に加えて,姿勢の回転方向を含む,計6つの自由度を持った情報である.センサの姿勢は,$z,y,x$軸周りの回転である3つのオイラー角$\alpha, \beta, \gamma$の組み合わせ以外にも,次の回転行列$\mathbf{R}$で表すこともでき,本論文では後者を採用する.
% \begin{equation}
% 	\mathbf{R}=
% 	\mathbf{R}_{z}(\alpha) \mathbf{R}_{y}(\beta) \mathbf{R}_{x}(\gamma)=
% 	\left[\begin{array}{ccc}\cos \alpha & -\sin \alpha & 0 \\ \sin \alpha & \cos \alpha & 0 \\ 0 & 0 & 1\end{array}\right]
% 	\left[\begin{array}{ccc}\cos \beta & 0 & \sin \beta \\ 0 & 1 & 0 \\ -\sin \beta & 0 & \cos \beta\end{array}\right]
% 	\left[\begin{array}{ccc}1 & 0 & 0 \\ 0 & \cos \gamma & -\sin \gamma \\ 0 & \sin \gamma & \cos \gamma\end{array}\right]
% \end{equation}
% \par
% 続いて,空間座標系を自己位置$(\mathbf{t},  \mathbf{R})$に応じて定義する.時刻$t=l$の自己位置によって定まる空間座標系(以後,座標系$t=l$と呼ぶ)上での時刻$t=k$の自己位置を$(\mathbf{t}^{(l)}_k,  \mathbf{R}^{(l)}_k)$として,
% \begin{equation}
% 	\mathbf{t}^{(k)}_k=\left(\begin{array}{c}
% 		0\\0\\0
% 	\end{array}\right), \,  
% 	\mathbf{R}^{(k)}_k=\left[\begin{array}{ccc} 1 & 0 & 0 \\ 0 & 1 & 0 \\ 0 & 0 & 1\end{array}\right]
% \end{equation}
% となるように空間座標系を定義する.以上の条件のもと,センサがカメラである場合には特に,投影平面の右方向に$x$軸,下方向に$y$軸,投影平面を垂直に奥向きに貫く方向を$z$軸とする.
% この時,座標系$t=k$上の点
% $\mathbf{m}^{(k)}={(x^{(k)} , y^{(k)}, z^{(k)})}^\mathrm{T}$は
% ,同次座標表現された位置ベクトル$\tilde{\mathbf{m}}^{(k)}={(x^{(k)} , y^{(k)}, z^{(k)},1)}^\mathrm{T}$を用いて座標系$t=l$上で
% \begin{equation}\label{eq:trans}
% 	\tilde{\mathbf{m}}^{(l)}
% 	=
% 	\left(\begin{array}{c}
% 		x^{(l)}\\ y^{(l)}\\ z^{(l)} \\ 1
% 	\end{array}
% 	\right)
% 	=
% 	\left[\begin{array}{cc}
% 		\mathbf{R}^{(l)}_k & \mathbf{t}^{(l)}_k \\ \mathbf{0} & 1
% 	\end{array}
% 	\right]
% 	\tilde{\mathbf{m}}^{(k)}
% \end{equation}
% と表現できる.
% 以後,特に座標系$t=0$を絶対座標系とし,それに従い$\mathbf{m}^{(0)}$を絶対座標,$\mathbf{R}^{(0)}_k,\, \mathbf{t}^{(0)}_k$をまとめて絶対自己位置または単に自己位置と呼ぶ.
% %また,座標系$t=l$での時刻$t=k$の自己位置$(\mathbf{t}^{(l)}_k,  \mathbf{R}^{(l)}_k)$を座標系$t=l$に対する時刻$t=k$の相対自己位置と呼ぶ.
% \par

% \subsection{要素技術}\label{subsec:techniques}

% 本研究で提案手法,および比較手法を構築するために用いられる画像処理・推定技術を紹介する.
% 具体的には,ビデオ・セグメンテーション,オプティカルフローの推定,単眼深度推定である.
% これらの技術は近年の機械学習の著しい進歩を背景に,深層学習に基づく推定手法が精度・速度の面で高い成果を上げている.

% \subsubsection{ビデオ・セグメンテーション}\label{subsec:video_segmentation}

% セグメンテーションとは画像に映り込んだ物体の画素領域を分離する技術であり,コンピュータービジョン分野におけるトピックの一つである.
% セグメンテーションの出力の一つとして得られる,注目する物体と他の部分において画素値が異なる画像は,マスク画像と呼ばれる.
% その中でもビデオ・セグメンテーションとは,動画を入力としてセグメンテーションを行うタスクである.
% 一枚の画像をセグメンテーションするイメージ・セグメンテーションと異なり,ビデオ・セグメンテーションはフレーム間の物体の対応関係が与えられるという点で,画像内での物体の追跡を包含している.
% 物体同士の対応については,コーナーやエッジ,輪郭といった部分的な特徴を持った点(以下,特徴点)を抽出し,それらをもとに探索・追跡が行われ,物体が対応付けられる.

% ビデオ・セグメンテーションはVideo Object Segmentation(以下,VOS)とVideo Semantic Segmentation(以下,VSS)に分かれる.
% VOSとは動画内の何かしらの物体をカテゴリの指定なく抽出するタスクである.
% セグメンテーションされた物体が何であるかという物体認識は含まれない.
% VSSとはあらかじめ定義されていたカテゴリ分類に基づいて物体を抽出するタスクである.
% 歩行者や車といったカテゴリに基づいて学習が行われるため,未知のカテゴリを分類することはできない.
% そのため,セグメンテーションの学習データがそろっていない物体に対してビデオ・セグメンテーションを行う場合には,VOSが適している.

% VOSは推定の過程でどれだけ人の手が加わるかに基づいて,Automatic VOS(以下,AVOS),Semi-automatic VOS(以下,SVOS),Interactive VOS(以下,IVOS)に分類できる\cite{garcia2018survey}.
% \begin{description}
%    \item[・Automatic Video Object Segmentation]\mbox{}\\
%    教師なしVOSとも呼ばれる.人の手を加えることなく,動画内で支配的な物体を検出し自動的にセグメンテーションを行う.動画のみを入力とするため,どの物体に注目するかを指定することができない.
%    \item[・Semi-automatic Video object Segmentation]\mbox{}\\
%    半教師ありVOSとも呼ばれる.動画の最初の画像のみに対して,抽出したい部分がセグメンテーションされた初期マスク画像を,動画に加えて入力とする.以後の動画について,初期のマスク画像の情報をもとに自動的にセグメンテーションを行う.
%    セグメンテーションしたい部分を四角の枠で囲った画像を初期の入力として与える手法やも存在する.
%    \label{SVOS}
%    \item[・Interactive Video object Segmentation]\mbox{}\\
%    教師ありVOSとも呼ばれる.各画像の推定過程において人が補助を加えながらセグメンテーションを行う.セグメンテーションの精度が高くなる一方で,長時間人によって監視する必要がある.
% \end{description}
% %動画内において注目する部分を指定したく,また,人の手をできるだけかけない方法としては,SVOSが適している.

% SVOSの関する研究では,2016年以前はセグメンテーション全般に使えるような特徴を学習し,それらの一般的な特徴と与えられた初期マスクから得た特徴を元に,後の画像についてセグメンテーションしていくことが主流であった.
% 次に,伝播型と呼ばれる,一時刻前のセグメンテーションで得られた画像を元に次のマスクを推定する手法が主流となった.しかし,伝播型のセグメンテーション手法は,推定が進むにつれてずれが蓄積されていくことや,注目している物体が他の物体によって遮られること(以下,オクルージョン)による推定精度の低下が課題となった\cite{garcia2018survey}.

% それに対して,2020年頃からSpace-Time Memory (STM)\cite{oh2019video}を用いて以前に計算されたセグメンテーション情報を格納しておき,注目している物体の時間的な変化を学習するモデルが主流となった.しかし,STMを用いた手法ではビデオが長尺になるほどメモリが不足するという問題が生じた.
% そこで,セグメンテーション情報を捉える際に,時間軸が異なるメモリを複数併用することで,長尺でもメモリ不足に陥らない手法が提案されている\cite{cheng2022xmem}.

% \subsubsection{オプティカルフロー}\label{subsec:optical_flow}

% オプティカルフロー(Optical Flow)推定とは,連続した2フレームの画像間で対応する点のフロー(画像座標上の変位)を推定するタスクである.
% 例えば,時刻$t=k$において画像座標$\mathbf{p}_k={(u_k,v_k)}^\mathrm{T}$に存在していた画素
% が,時刻$t=k+1$にける画像座標${\mathbf{p}'}_{k+1}={( {u'}_{k+1}, {v'}_{k+1})}^\mathrm{T}$に存在する画素
% に対応する場合に,この画素$\mathbf{p}_k$のフロー$\mathbf{f}(\mathbf{p}_k)=(f_{(u)},f_{(v)})^\mathrm{T}$は,
% \begin{equation}
%     \mathbf{f}(\mathbf{p}_k)=
% \left(
% \begin{array}{c}
%      f_{(u)}  \\
%      f_{(v)} 
% \end{array}
% \right)
% =
% \left(
% \begin{array}{c}
%      u'-u  \\
%      v'-v
% \end{array}
% \right)
% ={\mathbf{p'}}_{k+1}-\mathbf{p}_k
% \end{equation}
% と定義される.

% オプティカルフロー推定は疎なオプティカルフロー(Sparse Optical Flow)と密なオプティカルフロー(Dense Optical Flow)に大別される.
% 前者は抽出された特徴点についてのみフローを取得するのに対し,後者は全画素に対してフローを推定する.
% 密なオプティカルフローの出力として得られるフローのマップを補間することで,任意の画像座標のフローを推定することが可能になる.

% オプティカルフローは従来,人の手によって設定された損失関数をもとに,最適化問題として解かれていた
% \cite{chen2016full,horn1981determining,zach2007duality}ため,カメラや物体の推定速度や外的条件が変化した場合の安定性(以下,ロバスト性)に問題があった.
% 近年は,
% 2フレーム間から得た特徴を比較する畳み込みニューラルネットワーク(Convolutional Neural Network 以下,CNN)を導入したFlowNet\cite{dosovitskiy2015flownet}を皮切りに,深層学習に基づいて直接オプティカルフローを推定する手法が主流となっており,速度・精度・ロバスト性の面で従来手法を上回っている.

% \subsubsection{単眼深度推定}\label{subsec:depth_estimation}

% コンピュータービジョンの分野における深度推定とは,画像の各画素に対応する三次元空間上の点とカメラとのカメラ中心軸方向の距離(以下,深度)を推定する技術である.
% 絶対座標$\mathbf{m}^{(0)}$をもつ三次元空間上の点の時刻$t=k$における深度は,$m^{(k)}_{(z)}$にあたる.
% 深度推定の中でも,単眼カメラで捉えた画像一枚のみを入力とする単眼深度推定は,幾何学的に解くことが不可能な問題である.

% そこで近年,深層学習を用いて単眼カメラで撮影された画像のみから深度推定を行う手法が盛んに研究されてきた.
% CNNを用いた初期の深度推定手法はEigenら\cite{eigen2014depth}によって提案された.
% 単眼深度推定の教師あり学習は教師データとして画像に対応する密な(画素単位の)深度マップが必要であり,データセット作成の難しさから学習に必要な多くのデータセットを作れないことが課題であった.
% そこでステレオカメラで得た画像ペアを学習データとする半教師あり手法\cite{garg2016unsupervised}や,単眼カメラの動画を学習データとする教師なし手法\cite{zhou2017unsupervised}に注目が集まっている.
% 半教師あり学習の中には,教師あり学習と同程度の精度を誇るアーキテクチャ\cite{Guizilini_2020_CVPR}も存在する.

% しかし,深層学習をベースとした単眼深度推定は,学習するデータによって結果に差が出やすいという問題点がある.
% 特に,学習していない環境のデータに対しては深度のスケールを推定することができす,相対的な深度(以下,相対深度)のみしか求めることができない.
% この場合,数点の絶対深度を既知として与えることで,単眼深度推定によって得られた相対深度を絶対深度に変えるスケールの調整を行う必要がある\cite{ranftl2020towards}.
% また,学習済みのデータに対しても,物体までの距離に対して1割ほどの推定誤差がでてしまう\cite{fu2018deep}など,精度も課題である.

% \subsection{Visual SLAM}\label{subsec:visual_SLAM}

% 移動するセンサに対して,センサの位置姿勢と周辺の環境地図構築を同時推定する問題は,Simultaneous Localization and Mapping (以下,SLAM)と呼ばれている.
% SLAMは自律移動するあらゆる機械の制御に欠かすことができない技術であり,自動運転やドローンの自律飛行を目的として,近年活発に研究されている\cite{tomono正裕2018slam}.
% SLAMは以下のような手順で行われる.
% まず,各時刻にでのランドマーク(センサでとらえた特徴的な物体)とセンサの自己位置の幾何学的条件を複数取得する.
% これらの条件と,以前の時刻で推定された自己位置・周辺の環境地図から得られる条件を組み合わせ,現在の時刻の自己位置の推定と周辺の環境地図の更新・拡大を同時に行う.
% この作業を逐次的に繰り返すことで,各時刻のセンサの自己位置とセンサ周辺の環境地図を推定することが可能になる.

% SLAMの課題として,周辺環境が一様な場合にでは,情報量が足りずSLAMを構築できない,もしくは精度が低下する「退化」と呼ばれる現象が挙げられる\cite{tomono正裕2020}.退化の例として,白い壁が続く廊下でSLAMを行うと,レーザーまたはカメラからの入力だけではセンサが静止しているか移動しているかを判断できず,精度が低下する.
% これを克服するために,角速度を測定するジャイロセンサや,加速度センサ,全球測位衛星システム(Global Navigation Satellite System. 以下,GNSS)といった複数のセンサを融合させる手法も提案されている.

% SLAMは用いる入力センサの違いにより,LiDAR SLAMとVisual SLAMに大別される\cite{tomono正裕2020}.
% Visual SLAMについては続く項で取り上げる.

% LiDAR SLAMとは,LiDARと呼ばれるレーザスキャナを入力センサとしたSLAMである.
% LiDARは距離を直接計測して点群データを得られるため,安定したSLAMを実現しやすい.
% また,暗所でも用いることができる.
% その一方で,空間解像度が低く疎な点群しか得られないこと,時間解像度が低いため,動的物体が周辺環境に含まれる際に,それらを上手く捉えられないことなどが問題である.
% また,機器自体が高価であることも普及しづらい原因となっている\cite{tomono正裕2020}.

% \begin{figure}[b]
%     \centering
%     \includegraphics[width=0.5\linewidth]{images/SfM_2aspects.jpg}
%     \caption[Structure from Motionで解く三次元復元と視点推定の問題]{Structure from Motionで解く三次元復元(左)と視点推定(右)の問題(\cite{sfm_shalaby2017algorithms}より)}
%     \label{fig:sfm_figure}
% \end{figure}

% \subsubsection{Visual SLAMの特徴}\label{subsec:character_of_visual_SLAM}
% カメラを入力センサとして用いたSLAMはVisual SLAMと呼ばれている.
% カメラは,普及率が高く計測も容易なため,比較的扱いやすいセンサといえる.
% そのほかにも利点として,空間解像度・時間解像度が高いことがあげられる.
% 空間解像度が高いことにより微小な部分まで撮影対象をとらえることができ,時間解像度が高いことにより物体の高速な運動をとらえることが可能である.

% その一方で,画像自体には深度情報が含まれないため,Structure from Motion(以下,SfM)と呼ばれる三次元復元手法を組み込む必要があり,これがVisual SLAMを不安定化させる原因となっている\cite{tomono正裕2020}.
% SfM(図\ref{fig:sfm_figure})とはある対象を複数の視点から撮影した画像群を入力として,画像間の特徴点を対応づけることで撮影対象物の三次元復元と各視点の姿勢を同時推定する技術である\cite{中村恭之2017-08-05}.
% SfMでは,まず入力画像から抽出された特徴点を画像間で対応付け,それぞれの特徴点について次の方程式を立てる.

% 例えば,フレーム$t=k$上の特徴点$i$について,特徴点の画像座標(投影平面上右向きに$u$軸,下向きに$v$軸をとる)
% $\mathbf{p}_{k,i}={(u_{k,i} , v_{k,i})}^\mathrm{T}$と,
% 特徴点の三次元座標(座標系$t=k$上)
% $\mathbf{m}^{(k)}_{k,i}={(x^{(k)}_{k,i} , y^{(k)}_{k,i} , z^{(k)}_{k,i})}^\mathrm{T}$を用いて,
% \begin{equation}\label{eq:point2pro}
% 	\mathbf{p}_{k,i}
% 	=
% 	\boldsymbol{\pi}_{\text{Cam},k}
% 	\left(\mathbf{m}^{(k)}_{k,i}
% 	\right)
% \end{equation}
% と表現できる.ただし,フレーム$t=k$でのカメラの投影関数を次の$\boldsymbol{\pi}_{\text{Cam},k}$とした.
% ここで,$f_{x,k}$,$f_{y,k}$はフレーム$t=k$での焦点距離,$(c_{x,k},c_{y,k})$はフレーム$t=k$での光学中心の画像座標を指す.
% 入力画像がすべて同じカメラで撮影されている場合,この投影関数$\boldsymbol{\pi}_{\text{Cam},k}$は全フレームで共通のものとなる.
% \begin{equation}\label{eq:projection}
% 	\boldsymbol{\pi}_{\text{Cam},k}
% 	\left(
% 	\begin{array}{c}
% 	X\\Y\\Z
% 	\end{array}
% 	\right)
% 	=
% 	\left(
% 	\begin{array}{c}
% 		\frac{X}{Z}f_{x,k}+c_{x,k}  \\ \\  \frac{Y}{Z}f_{y,k}+c_{y,k} 
% 	\end{array}
% 	\right)
% \end{equation}
% このとき,$t=k$の自己位置$\mathbf{R}^{(0)}_{k}, \, \mathbf{t}^{(0)}_{k} $を用いて,
% フレーム$t=k$上の特徴点$i$の三次元同次絶対座標は
% \begin{equation}\label{eq:henkan}
%     \tilde{\mathbf{m}}^{(0)}_{k,i}=
%     \left(\begin{array}{cc}
% \mathbf{R}^{(0)}_{k}  &  \mathbf{t}^{(0)}_{k}  \\
%  \mathbf{0} & 1 
% \end{array}\right)
%     \tilde{\mathbf{m}}^{(k)}_{k,i}
%    % = \mathbf{X}^{(0)}_{k}\,\mathbf{m}^{(k)}_{k,i} 
% \end{equation}
% より得られる.
% これがすべてのフレームにおいて同時に成立することから,方程式の未知数,すなわち,各フレームの自己位置
% $ \mathbf{R}^{(0)}_{k}, \, \mathbf{t}^{(0)}_{(k)}$
% と各特徴点の三次元座標$\mathbf{m}^{(0)}_{k}$を推定する.

% Visual SLAMでは以前のフレームと現在のフレームを用いてSfMを行うことにより,現在のフレームの自己位置および周辺環境の三次元地図を逐次的に更新していく.

% \subsubsection{動的環境下でのVisual SLAM}\label{subsec:dynamic_visual_SLAM}
% ここ数十年でVisual SLAMは急激な発展を遂げてきたが,ほとんどの技術は周辺環境が変化しないという想定で成り立つものであった.
% Visual SLAMはSfMを基礎技術として用いており,SfMは撮影対象物が剛体かつ静止していることを前提としているからである.
% そのため,動的物体を含む環境に対しては,特徴点の誤対応や,オクルージョンにより精度が急激に落ちるという問題点があった\cite{tan2013robust}.
% これを克服したのが,動的物体と静止背景を個別に扱い,動的物体の影響を排除する方法である\cite{saputra2018visual}.
% 動的物体の影響を排除する方法には幾何学的条件を用いるといった手法も挙げられるが,\ref{subsec:video_segmentation}節で説明したセグメンテーションを用いる手法は具体的に以下の手順を踏む.
% \begin{enumerate}
%  \item 入力された連続画像に対してセグメンテーションを行うことで,動的物体の画素領域と静止背景の画素領域に分類する(図\ref{fig:segmentation_semantic},\ref{fig:segmentation_mask}).
%  \item 各時刻の画像から検出された特徴点のうち,静止背景の画素領域のみから特徴点を抽出し(図\ref{fig:extract}),それらの静止している特徴点を用いて従来のVisual SLAMを適用する.これにより,カメラ姿勢推定と静止背景のみの三次元環境地図の構築を行う.
%  \label{tejun2}
%  \item 上の手順で得られたカメラ姿勢とセグメンテーションされた動的物体の深度推定結果などを統合し,動的物体の周辺環境地図内での三次元復元および追跡を行う.
% \end{enumerate}

% \begin{figure}[h]
% \centering
% 	\begin{minipage}[t]{0.45\hsize}
% 		\centering
% 		\includegraphics[width=0.8\linewidth]{images/segmentation_semantic.png}
% 		\subcaption{セグメンテーションによって得られたラベル付き領域}
% 		\label{fig:segmentation_semantic}
%  	\end{minipage}
% 	\begin{minipage}[t]{0.45\hsize}
% 		\centering
% 		\includegraphics[width=0.8\linewidth]{images/segmentation_mask.png}
% 		\subcaption{動的物体(人)の領域を抽出したマスク}
% 		\label{fig:segmentation_mask}
% 	\end{minipage}
% 	\begin{minipage}[t]{0.45\hsize}
% 		\centering
% 		\includegraphics[width=0.8\linewidth]{images/segmentation_extracted.png}
% 		\subcaption{マスクにより分離された背景の特徴点(青)と動的物体上の特徴点(赤)の例}
% 		\label{fig:extract}
		
% 	\end{minipage}
%  \caption[動的物体のセグメンテーション,
%  特徴点分離の例]{動的物体のセグメンテーション,
%  	特徴点分離の例(\cite{liu2021rds}より)}
%  \label{fig:segmentation}
% \end{figure}

% 動的物体を含む環境下でのVisual SLAMに関する研究の多くは,上記の手順\ref{tejun2}までのカメラの姿勢推定および静的な背景の復元に重きを置いてきた.
% 動的物体の部分についても,動的物体が直線や円錐曲線上を動くと仮定して追跡する\cite{alcantarilla2012combining,avidan1999trajectory}など,動きに対して何かしらの仮定が設けられている.
% 動的物体の複雑な動きを捉えることや,高精度な位置推定・表面の微小な変位推定を行おうとしたVisual SLAMの研究は少ない.



% \clearpage

%----------第3章----------
\section{提案手法の構築}\label{sec:method_creation}
本章では,本論文の提案手法を詳述する.

\subsection{手法の概要}\label{subsec:method_outline}
本節では,本論文のベースラインとなる手法と提案手法の概要を説明する.
本論文ではDeblurring3DGS\cite{Deblurring3dgs}をベースラインとして用いる.
提案手法として,ベースラインのピンボケの表現に,カメラの原理に基づくピンボケ表現の導入をする手法と,カメラ標定要素修正を伴う最適化を追加した手法の二手法を用いる.
\subsubsection*{ベースライン}
本論文でベースラインとなるDeblurring3DGS\cite{Deblurring3dgs}は,ピンボケをガウシアンの変形として扱い,その変形分をMLPを用いて推定することでピンボケを表現する3DGS手法である.
MLP$\mathcal{F}_\theta$は各ガウシアン$j$の位置$x_j$, クォータニオン$r_j$, スケール$s_j$と, レンダリングするカメラの位置$v$を入力として,ガウシアンのクォータニオンの変化分$\delta_{r_j}$とスケールの変化分$\delta_{s_j}$を出力する.
なお,クォータニオン$r$とスケール$s$は,ガウシアンの共分散行列$\Sigma$を用いて以下のように表せる.
\begin{equation}
  \Sigma(r, s) = R(r)S(s){S(s)}^T{R(r)}^T
\end{equation}
ここで,$R(r)$はクォータニオン$r$に対応する回転行列,$S(r)$はスケール$s$に対応するスケーリング行列である\cite{rotation_quaternion}.
得られたクォータニオンの変化分$\delta_{r_j}$とスケールの変化分$\delta_{s_j}$を用いて,以下のようにガウシアン$j$のクォータニオン$r_j$とスケール$s_j$をそれぞれ変化させる.
\begin{equation}
	\hat{r}_j=r_j \cdot \min \left(1.0, \lambda_s \delta r_j+\left(1-\lambda_s\right)\right), 
	\hat{s}_j=s_j \cdot \min \left(1.0, \lambda_s \delta s_j+\left(1-\lambda_s\right)\right)
\end{equation}
ここで,$\lambda_s$は最適化の安定性のために導入される係数で,MLPの出力を$\lambda_s$拡大し,$1-\lambda_s$ずらす.
こうして得られたガウシアン$G(x_j, \hat{r}_j, \hat{s}_j)$は,カメラ位置$v$で撮影された画像でのピンボケを表現するよう変形されている.
ガウシアンの最適化では,変形したガウシアン$G(x_j, \hat{r}_j, \hat{s}_j)$をレンダリングした画像と入力であるピンボケ画像の差分を損失とする.
損失を最小化するように,ガウシアンのパラメタ$x_j$, $r_j$, $s_j$, $\alpha_j$, $c_j$に加えて,MLPのパラメタ$\theta$を更新する.
なお,$\alpha_j$は不透明度,$c_j$は色を表す球面調和関数である.
更新を繰り返すことで,ボケていない鮮明なガウシアン$G(x_j, r_j, s_j, \alpha_j, c_j)$と,ピンボケを表現するMLP$\mathcal{F}_\theta$を得る.

\subsection{幾何学的条件を考慮したピンボケ表現の導入}\label{subsec:introduction_of_blur_expression}
\subsubsection{ピンボケの原理}\label{subsubsec:principle_of_blur}
\subsubsection{深度を考慮した正則化項の追加}\label{subsubsec:add_regularization_term}
\subsection{カメラ標定要素修正を伴う三次元モデルの最適化}\label{subsec:pose_correction}
\subsubsection{空間変換を用いたガウシアンとカメラ標定要素の交互最適化}\label{subsubsec:sequential_optimization}
\subsubsection{重み付けによる修正量の調整}
\subsubsection*{閾値による修正の判別}
\subsubsection*{画像誤差を用いた学習率の調整}
\cite{3d_warping}
\clearpage

% %----------第4章----------
% \section{提案手法の適用}\label{sec:method_appliance}
本章では\ref{sec:method_creation}章で述べた手法を実データに対して適用し,評価する.
本研究では回転する扇風機の羽根を剛体運動をする動的物体とみなし,以下の三つの項目を推定,評価した.
\begin{enumerate}
	\item 動的物体上の各画像座標の画像上の変位\label{item:flow}
	\item 各時刻での動的物体上の各画像座標に対応する三次元位置\label{item:location}
	\item 動的物体上の各画像座標に対応する三次元空間内の変位\label{item:3d_trans}
\end{enumerate}
項目\ref{item:flow}では,オプティカルフロー推定の結果を用いて逐次的に画像上の追跡を行った場合の画像座標の推定精度を評価した.
項目\ref{item:location},\ref{item:3d_trans}については,提案手法である動的物体に対してSLAMを適用する方法と,比較手法である単眼深度推定を用いる方法の二手法で比較した.

\subsection{検証データについて}\label{subsec:data}
本節では検証に用いたデータの説明をする.

\subsubsection{入力データ}\label{subsubsec:input_data}
本研究では入力データとして,扇風機が回転する様子を静止したハイスピードカメラから撮影した連続RGB画像を用いた(図\ref{fig:input_data}).
合計200枚のRGB連続画像であり,この動画の中で扇風機の羽は二周弱回転する.
なお,元のデータはモノクロ画像であったものを,RGBすべての画素値を同じ値にすることでRGB画像とした.
フレームレートは2000フレーム毎秒であり,動画全体は実際の0.1秒間にあたる.
解像度は横1280×縦1048ピクセルである.
扇風機の羽根は,一枚が透明,その他の三枚の羽には透明な羽根から時計回りで順に,小,大,中,三種類の大きさのドットパターンが付与されている.
これらのドットパターンは後述するデジタル画像相関法に用いられた.
大きさは,羽部分の直径が約270 mm, 奥行きが最大70 mmとである.

\begin{figure}[h]
	\centering
	\includegraphics[width=0.8\linewidth]{images/chapter4/input_images.png}
	\caption[入力画像]{入力画像(全200枚)}
	\label{fig:input_data}
\end{figure}

\begin{comment}
\begin{figure}[h]
	\centering
	\begin{minipage}[t]{0.45\hsize}
		\centering
		
		\includegraphics[width=0.8\linewidth]{images/chapter4/input_data000.png}
		\subcaption{$t=0$}
		\label{fig:input_image0}
		
	\end{minipage}
	\begin{minipage}[t]{0.45\hsize}
		\centering
		
		\includegraphics[width=0.8\linewidth]{images/chapter4/input_images.png}
		\label{fig:input_images}
	\end{minipage}
	\caption[動的物体のセグメンテーション]{入力画像}
	\label{fig:input}
\end{figure}
\end{comment}

\subsubsection{正解データ}\label{subsubsec:correct_data}

本研究では正解データとして,扇風機の羽根上の表面の点の三次元座標を用いた(図\ref{fig:correct_points}).
これは二視点から撮られた連続画像を用いたデジタル画像相関法によって得られたものであり,全部で2550点分(フレームごとに欠損あり)のデータである.
データは各時刻を通して羽根上の共通の点を追跡しており,各点の三次元座標の時系列データとなっている.
正解データの測定は1000回毎秒であり,入力動画と対応する100フレーム(0.1秒)分のデータを用いた.
一番最初に正解データが取得された時刻を$t=0$とし,$t=0$から数えて$k$番目の正解データが取得された時刻を$t=k$と定義する.
すなわち,データは全部で$t=0$から$t=99$まであり,これが実際の0.000秒から0.099秒の間に取得されたデータとなる.
なお,正解データの三次元点を画像上に投影した点の画像座標を,本論文では正解データの画像座標と呼ぶ.

\begin{figure}[h]
	\centering
	\begin{minipage}[c]{0.7\hsize}
		\centering
		\includegraphics[width=0.8\linewidth]{images/chapter4/correct_points.png}
		\subcaption{正解データを可視化した様子}
		\label{fig:correct_points3D}
		
	\end{minipage}
	\begin{minipage}[c]{0.25\hsize}
		\centering
		
		\includegraphics[width=1\linewidth]{images/chapter4/image_plot.png}
		\subcaption{画像上に投影された正解データ(緑色の点)}
		\label{fig:plot_image}
	\end{minipage}
	\caption[正解データの様子]{正解データの様子}
	\label{fig:correct_points}
\end{figure}

\subsection{セグメンテーションの適用}\label{subsec:segmentation_eval}

本項では画像内で動的物体を分離するために用いたセグメンテーションの結果を示す.
本研究では扇風機の4枚の羽根のうち透明でない3枚の羽根を,剛体運動をする一つの動的物体とみなしセグメンテーションを行った.
初期マスクとして入力動画の1枚目の画像と対応する図\ref{fig:seg000}のようなマスクを与え,それ以降の200枚の画像に対してセグメンテーションを行った.
セグメンテーションの結果の一部を次ページに示す(図\ref{fig:segmentaiton_output}).
全体として,扇風機の羽根という未知の物体に対しても,200フレームという長尺にわたってセグメンテーションが安定して行えていた.
一部,図\ref{fig:seg050}や図\ref{fig:seg150}の大きなドットパターンの羽根(左側の羽根)の内側部分のように,初期マスクと一致していない部分が確認された.
これは,大きなドットパターンの羽根が画面左側にある際に,内側部分がカメラから見て中心の回転軸の奥側に隠れてしまうことによって生じる誤対応と考えられる.
また,初期マスクでは含まれていなかった羽根外側の白いポイント状の部分がマスクに含まれてしまう事象(\ref{fig:seg075}の右の羽根,\ref{fig:seg125}の上の羽根など)も発生した.
これは羽根の境界部分は白っぽい色をしているため,白いポイントの部分を境界の内部と判定したためだと考えられる.

\newpage

\begin{figure}[H]
	\centering
	\begin{minipage}[t]{0.4\hsize}
		\centering
		\includegraphics[width=0.9\linewidth]{images/chapter4/segmentation/image-000000.png}
		\subcaption{0枚目(入力)}
		\label{fig:seg000}
	\end{minipage}
	\begin{minipage}[t]{0.4\hsize}
		\centering
		\includegraphics[width=0.9\linewidth]{images/chapter4/segmentation/image-000025.png}
		\subcaption{25枚目}
		\label{fig:seg025}
	\end{minipage}
	\begin{minipage}[t]{0.4\hsize}
		\centering
		\includegraphics[width=0.9\linewidth]{images/chapter4/segmentation/image-000050.png}
		\subcaption{50枚目}
		\label{fig:seg050}
	\end{minipage}
	\begin{minipage}[t]{0.4\hsize}
		\centering
		\includegraphics[width=0.9\linewidth]{images/chapter4/segmentation/image-000075.png}
		\subcaption{75枚目}
		\label{fig:seg075}
	\end{minipage}
	\begin{minipage}[t]{0.4\hsize}
		\centering
		\includegraphics[width=0.9\linewidth]{images/chapter4/segmentation/image-000100.png}
		\subcaption{100枚目}
		\label{fig:seg100}
	\end{minipage}
	\begin{minipage}[t]{0.4\hsize}
		\centering
		\includegraphics[width=0.9\linewidth]{images/chapter4/segmentation/image-000125.png}
		\subcaption{125枚目}
		\label{fig:seg125}
	\end{minipage}
	\begin{minipage}[t]{0.4\hsize}
		\centering
		\includegraphics[width=0.9\linewidth]{images/chapter4/segmentation/image-000150.png}
		\subcaption{150枚目}
		\label{fig:seg150}
	\end{minipage}
	\begin{minipage}[t]{0.4\hsize}
		\centering
		\includegraphics[width=0.9\linewidth]{images/chapter4/segmentation/image-000175.png}
		\subcaption{175枚目}
		\label{fig:seg175}
	\end{minipage}
	\caption[セグメンテーションの結果]{セグメンテーションの結果.入力画像に出力されたマスク(赤色)を重ねた.}
	\label{fig:segmentaiton_output}
\end{figure}

\subsection{オプティカルフロー推定の精度評価}\label{subsec:opticalflow_eval}
ここでは,比較手法に用いるために前処理として行った,オプティカルフロー推定の適用結果および精度評価を示す.
なお,オプティカルフロー推定は正解データに対応する時刻の画像100枚を入力とした.
\subsubsection*{適用結果}
オプティカルフロー推定の適用結果は図\ref{fig:optical_flow_output}のようになった.
各時刻で,羽根の部分が回転している様子はきれいにとらえられている.
一方でフローを正確に推定できていない部分としては,図\ref{fig:flow0}の上部や図\ref{fig:flow20}の右上部分のような扇風機よりも外側の部分,扇風機の回転軸などがあげられる.
これらの部分は色の変化や輝度の変化が少ないことが原因と考えられる.
実際に,透明な羽根上にある黒で囲まれた白い丸の部分は図\ref{fig:flow30}のように動きが認識されている.
また,透明な羽根の部分についても止まっていると認識されている.
これは,透明な羽根の奥側の部分を捉えているためだといえる.
各羽根のドットパターンの大小の違いによるオプティカルフロー推定結果の変化は確認されなかった.

\begin{figure}[H]
	\centering
	\begin{minipage}[t]{0.4\hsize}
		\centering
		\includegraphics[width=0.9\linewidth]{images/chapter4/optical_flow/flowt0.png}
		\subcaption{$t=0$}
		\label{fig:flow0}
	\end{minipage}
	\begin{minipage}[t]{0.4\hsize}
		\centering
		\includegraphics[width=0.9\linewidth]{images/chapter4/optical_flow/flowt10.png}
		\subcaption{$t=10$}
		\label{fig:flow10}
	\end{minipage}
	\begin{minipage}[t]{0.4\hsize}
		\centering
		\includegraphics[width=0.9\linewidth]{images/chapter4/optical_flow/flowt20.png}
		\subcaption{$t=20$}
		\label{fig:flow20}
	\end{minipage}
	\begin{minipage}[t]{0.4\hsize}
		\centering
		\includegraphics[width=0.9\linewidth]{images/chapter4/optical_flow/flowt30.png}
		\subcaption{$t=30$}
		\label{fig:flow30}
	\end{minipage}
	\begin{minipage}[t]{0.4\hsize}
		\centering
		\includegraphics[width=0.9\linewidth]{images/chapter4/optical_flow/flowt40.png}
		\subcaption{$t=40$}
		\label{fig:flow40}
	\end{minipage}
	\begin{minipage}[t]{0.4\hsize}
		\centering
		\includegraphics[width=0.4\linewidth]{images/flow_legends.png}
		\subcaption{フローの凡例.フローの距離と向きにそれぞれ明度と色相が対応する.}
		\label{fig:flow_legends}
	\end{minipage}
	\caption[オプティカルフロー推定の結果]{オプティカルフロー推定の結果.各時刻の対応する画像(左)と出力(右)}
	\label{fig:optical_flow_output}
\end{figure}

\subsubsection*{精度評価}
本研究では画像上の位置の評価として,各時刻における正解データの画像座標$\mathbf{p}^{*}_{i}$と,オプティカルフローにより推定された座標$\mathbf{p}_i$の誤差を以下の二指標で評価した.
正解データは羽根上に分布しているため,羽根の部分以外のフローの評価は行えていない.
なお,データ数$N$は各時刻の評価に用いた点の数,$N_{in}$は推定された点の画像座標が正解点の半径1画素以内にある点の合計数である.
\begin{itemize}
	\item Endpoint Error(EE) : $\displaystyle \frac{1}{|{N}|} \sum_{{i} \in {N}}\left\|\mathbf{p}_{{i}}-\mathbf{p}_i^*\right\|$
	\\
	\\
	\item 一画素以内の割合 : $\displaystyle \frac{|{N_{in}}|}{|{N}|}$
\end{itemize}
まず各時刻のオプティカルフロー推定の精度評価として,$t\geqq20$の各時刻のデータに対して,一時刻先の推定画像座標の精度評価を行った結果を図\ref{fig:of_eval0}に示す.
全時刻の平均はEEが0.155308 ピクセル,半径一画素以内に収まる割合が0.999228 であった.
羽根の内部では非常に高い精度でオプティカルフロー推定されているといえる.
一方で,二指標ともに悪い結果となった時刻$t=51$では,大きなドットパターンの羽根の境界部分でオプティカルフローの推定がうまくいっておらず,推定点と正解点の誤差が大きくなっていることがわかる(図\ref{fig:t51}).

\begin{figure}[H]
	\centering
	\begin{minipage}[b]{0.45\hsize}
		\centering
		\includegraphics[width=0.9\linewidth]{images/chapter4/flow/ee0.png}
		\subcaption{EE}
		\label{fig:ee0}
	\end{minipage}
	\begin{minipage}[b]{0.45\hsize}
		\centering
		\includegraphics[width=0.9\linewidth]{images/chapter4/flow/1gaso0.png}
		\subcaption{一画素以内の割合}
		\label{fig:1gaso0}
	\end{minipage}
	\caption{一時刻先のオプティカルフロー推定の精度}
	\label{fig:of_eval0}
\end{figure}
\begin{figure}[H]
	\centering
	\includegraphics[width=0.5\linewidth]{images/chapter4/flow/t51_2.png}
	\caption[時刻$t=51$での推定点と正解点]{時刻$t=51$での推定点(緑)と正解点(赤)の様子}
	\label{fig:t51}
\end{figure}

次に,画像上の追跡の精度評価として,$t=51$の正解データ画像座標を初期値として与え,式\ref{eq:tracking}にもとづいて各点の追跡を行った場合の精度を示す.
なお,ここで推定された各時刻の画像座標は,比較手法である単眼深度推定を用いた手法の追跡精度を評価する際に使用した.
オプティカルフローの推定結果を用いて逐次的に各点の位置を画像上で推定した場合の精度は図\ref{fig:of_eval1}のようになった.
時間がたつにつれてオプティカルフロー推定の誤差が蓄積されていることがわかる.
最後のフレームである$t=99$においては,EEが2.1 ピクセル,一画素以内の割合が0.232となった.
半径一画素以内の割合は,$t=55$で0.973であったのに対し,その5フレーム後の$t=63$には0.642になるなど,十数フレームの間に急激に低下している.
$t=80$付近から$t=90$付近にかけてEEが減少,一画素以内の割合が上昇しているのは,小さなドットパターンの羽根で正解データの画像座標に近づく点が増えたためだと確認できる(図\ref{fig:t80-90}).

\begin{figure}[H]
	\centering
	\begin{minipage}[b]{0.45\hsize}
		\centering
		\includegraphics[width=0.9\linewidth]{images/chapter4/flow/ee2.png}
		\subcaption{EE}
		\label{fig:ee2}
	\end{minipage}
	\begin{minipage}[b]{0.45\hsize}
		\centering
		\includegraphics[width=0.9\linewidth]{images/chapter4/flow/1gaso2.png}
		\subcaption{一画素以内の割合}
		\label{fig:1gaso2}
	\end{minipage}
	\caption{画像上の追跡の精度}
	\label{fig:of_eval1}
\end{figure}

\begin{figure}[H]
	\centering
	\begin{minipage}[b]{0.45\hsize}
		\centering
		\includegraphics[width=0.7\linewidth]{images/chapter4/flow/image_80_trimmed.png}
		\label{fig:t80}
	\end{minipage}
	\begin{minipage}[b]{0.45\hsize}
		\includegraphics[width=0.7\linewidth]{images/chapter4/flow/image_90_trimmed.png}
		\label{fig:t90}
	\end{minipage}
	\centering
	\caption[$t=80$と$t=90$における誤差一画素以内の点の変化]{$t=80$(左)と$t=90$(右)における誤差一画素以内の点の変化.緑色の点が,推定点と正解点の誤差が1画素以内の点.}
	\label{fig:t80-90}
\end{figure}

\subsection{動的物体の三次元位置座標推定の精度評価}\label{subsec:position_eval}
本節では動的物体に対してSLAMを行う提案手法の適用結果を示したのち,各時刻で正解データの画像座標を与えた場合の比較手法と提案手法の三次元位置座標推定の精度を比較する.
評価はSLAMの生成する三次元点群が安定する時刻$t\geqq{20}$で行った.
各時刻で画像上に投影された正解データの各点を二つの手法でそれぞれ三次元再構成し,正解データの三次元位置と推定点の三次元位置の誤差で評価した.

\subsubsection*{動的物体に対するSLAMの適用}
ここではセグメンテーションで得た動的物体のマスクを用いて,動的物体に対してSLAMを適用した結果を示す.
Visual SLAMは数多くのアーキテクチャが存在しているが,本研究ではMur-Artalら\cite{mur2017orb}のORB-SLAM2をベースに手法を構築した.
入力となるRGB画像から特徴点を検出し,その中からマスクを用いて動的物体の画素領域上にある特徴点のみを抽出した(図\ref{fig:feature_image}).
それらの特徴点を用いてSLAMを行い,自己位置として扇風機の羽根に対するカメラの各時刻の相対自己位置を,周辺環境地図として扇風機の羽根の表面の三次元点群を得た(図\ref{fig:slam_output})
SLAMにより得られた相対自己位置および三次元点群は実空間とスケールがあっていないため,手作業により得られた座標空間の値を定数倍することでスケールを調整した.
\begin{figure}[h]
	\centering
	\begin{minipage}[b]{0.45\hsize}
		\centering
		\includegraphics[width=0.8\linewidth]{images/chapter4/feature_image.png}
		\subcaption[マスクにより抽出された特徴点と排除された特徴点]{マスクにより抽出された特徴点(緑)と排除された特徴点(赤)}
		\label{fig:feature_image}
	\end{minipage}
	\begin{minipage}[b]{0.45\hsize}
		\centering
		\includegraphics[width=0.9\linewidth]{images/chapter4/slam_output1.png}
		\subcaption[SLAMの出力の様子]{SLAMの出力の様子.緑の四角い枠がカメラの相対自己位置,黒い点群が羽根の三次元点群}
		\label{fig:slam_output}
	\end{minipage}
	\caption[動的物体に対するSLAMの適用]{動的物体に対するSLAMの適用}
	\label{fig:apply_SLAM}
\end{figure}
\subsubsection*{三次元位置の評価指標}
本研究では三次元位置の評価として,各時刻における正解データの三次元座標$\mathbf{r}^{*}_{i}$と,推定された三次元座標$\mathbf{r}_i$の誤差を以下の三指標で評価した.
なお,データ数$N$は各時刻における評価に用いた点の数である.
\begin{itemize}
	\item Relative Absolute Error(Abs Rel):$\displaystyle \frac{1}{|N|} \sum_{i \in N} \frac{\left\|r_i-r_i^*\right\|}{\left\|r_i^*\right\|}$
	\\
	\\
	\item Mean Absolute Error(MAE):$\displaystyle \frac{1}{|{N}|} \sum_{{i} \in {N}}\left\|r_{{i}}-r_i^*\right\|$
	\\
	\\
	\item Root Mean Square Error(RMSE):$\displaystyle \sqrt{\frac{1}{|{N}|} \sum_{{i} \in {N}}\left\|r_{{i}}-r_i^*\right\|^2}$
	
\end{itemize}
Abs Relは推定値と真値の距離の差を真値までの距離で除した値の平均であり,評価する点群のスケールによらない値である.
MAE,RMSEはどちらも推定値と真値の距離の差をもとに平均化した値であり,扱う点群スケールによって大きさが左右される.

\subsubsection{比較手法の位置座標推定の精度評価}
単眼深度推定を用いた手法の各時刻の結果を図\ref{fig:monodepth_position_result}に示す.
各指標の全時刻での平均は,Abs Relが0.0199,MAEが17.825 mm, RMSEが22.166 mm となった.
Abs Relに関しては,カメラから1 m離れた物体に対して約20 cmの誤差が生じるほどということになる.
三指標のグラフは同様な形状をしており,約50フレームおきに周期性を持っている.
これは,羽根が約50フレームで一周するため,約50フレームおきに同じような画像が入力になったことが原因と考えられる.
また,いずれの指標でも$t=55$で大きな値をとっている.
これは,スケール調整の際に絶対深度を与えた点(全部で5つ)の推定相対深度の値が近かったことにより,スケールパラメタ$\omega_1$が大きな値をとったことが原因と考えられる.
実際に三次元空間上に推定点をプロットした様子を見ると,奥行き方向に推定点が広く散らばっている(図\ref{fig:depth_position_t55}).
各指標が小さな値をとっている$t=90$では,推定された点が一つの深度付近に分布していることがわかる(図\ref{fig:depth_position_t90}).
これは,スケール調整の際に絶対深度を与えた点の推定相対深度の値が遠かったことにより,スケールパラメタ$\omega_1$が小さくなり,シフトパラメタ$\omega_0$付近に推定された絶対深度が集中したためと考えられる.
単眼深度推定を用いた手法では羽根の深度を正確には推定できておらず,相対深度から絶対深度にスケールを調整する際のパラメタの値の違いにより,誤差の値も大きく変わってくるといえる.

\begin{figure}[h]
	\centering
	\begin{minipage}[b]{0.45\hsize}
		\centering
		\includegraphics[width=0.8\linewidth]{images/chapter4/position/depth_position_t55.png}
		\subcaption{$t=55$の様子}
		\label{fig:depth_position_t55}
	\end{minipage}
	\begin{minipage}[b]{0.45\hsize}
		\centering
		
		\includegraphics[width=0.8\linewidth]{images/chapter4/position/depth_position_t90.png}
		\subcaption{$t=90$の様子}
		\label{fig:depth_position_t90}
	\end{minipage}
	\caption[単眼深度推定を用いた手法における特徴的な時刻の様子]{単眼深度推定を用いた手法における特徴的な時刻の様子.白色が正解点,緑色が推定された点.}
	\label{fig:monodepth_position2}
\end{figure}


\begin{figure}[H]
	\centering
	\begin{minipage}[t]{0.45\linewidth}
		\centering
		\includegraphics[width=0.9\linewidth]{images/chapter4/position/absrel1.png}
		\subcaption{Abs Rel}
		\label{fig:absrel1}
	\end{minipage}
	\begin{minipage}[t]{0.45\linewidth}
		\centering
		\includegraphics[width=0.9\linewidth]{images/chapter4/position/mae1.png}
		\subcaption{MAE}
		\label{fig:mae1}
	\end{minipage}
	\begin{minipage}[t]{0.45\linewidth}
		\centering
		\includegraphics[width=0.9\linewidth]{images/chapter4/position/rmse1.png}
		\subcaption{RMSE}
		\label{fig:rmse1}
	\end{minipage}
	\caption{単眼深度推定を用いる手法の位置座標推定の精度}
	\label{fig:monodepth_position_result}
\end{figure}

\newpage
\subsubsection{提案手法の位置座標推定の精度評価}\label{3d_position_eval_slam}
提案手法である,動的物体にSLAMを用いた手法の結果を以下に示す.
各指標の全時刻での平均は表\ref{table:compare_position}のようになった.
比較手法である単眼深度推定を用いた手法と比べ,すべての指標で90 \%以上の誤差の低減が確認された.
Abs Relに関しては,カメラから1 m離れた物体に対して約1 mmの誤差が生じる程度,MAEも1 mm以内と高い精度で正解データの三次元位置を復元できているといえる.

\begin{table}[H]
	\caption{各指標の全時刻での平均値}
	\label{table:compare_position}
	\centering
	\begin{tabular}{lcc}
		\hline
		評価指標  & 単眼深度推定を用いる手法  &  SLAMを用いる手法  \\
		\hline \hline
		Abs Rel  & 0.0199 & 0.0011 \\
		\hline
		MAE (mm)  & 17.825 & 0.9446 \\
		\hline
		RMSE (mm)  & 22.166 & 1.6227 \\
		\hline
	\end{tabular}
\end{table}
\begin{comment}
\hline
評価指標  & 単眼深度推定を用いる手法  &  SLAMを用いる手法  \\
\hline \hline
Abs Rel  & 0.01992644 & 0.001059725 \\
\hline
MAE (mm)  & 17.825 & 0.9446875 \\
\hline
RMSE (mm)  & 22.16625 & 1.62275 \\
\hline
\end{comment}
各時刻の評価指標の推移を図\ref{fig:slam_position_result}に示す.
時刻ごとの評価指標の推移を見ても,三指標ともに平均値の付近を推移しており,数値のブレは単眼深度推定を用いる手法と比べると少ない.
三指標のグラフの形状は$t=80$付近を除いて同じような形状をしている.
Abs Rel, MAEともに最も低い値をとった$t=65$では高い精度で三次元座標を推定できていることが確認できる(図\ref{fig:slam_position2}).
RMSEが他の二指標と比べて大きくなっている$t=81$ではカメラに向かって左側の羽根の一部で,深度方向に大きな誤差が出ていることが確認できる(図\ref{fig:slam_position3}).
これはSLAMによって環境地図として生成された動的物体の疎な三次元点群が,外れ値となる点群を多く含んでいた影響といえる(図\ref{fig:slam_position4}).
それにより大きな誤差を持ったいくつかの推定点が,RMSEを大きく引き上げる結果となった.

\begin{figure}[h]
	\centering
	\begin{minipage}[b]{0.45\hsize}
		\centering
		\includegraphics[width=0.8\linewidth]{images/chapter4/position/slam_position_t65_1.png}
		\subcaption{右斜め横から見た様子}
		\label{fig:slam_position_t65_1}
	\end{minipage}
	\begin{minipage}[b]{0.45\hsize}
		\centering
		\includegraphics[width=0.8\linewidth]{images/chapter4/position/slam_position_t65_2.png}
		\subcaption{右の真横から見た様子}
		\label{fig:slam_position_t65_2}
	\end{minipage}
	\caption[SLAMを用いる手法における$t=65$の様子]{SLAMを用いる手法における$t=65$の様子.白色が正解点,緑色が推定された点.}
	\label{fig:slam_position2}
\end{figure}

\begin{figure}[h]
	\centering
	\begin{minipage}[b]{0.45\hsize}
		\centering
		\includegraphics[width=0.8\linewidth]{images/chapter4/position/slam_position_t81_1.png}
		\subcaption{左斜め横から見た様子}
		\label{fig:slam_position_t81_1}
	\end{minipage}
	\begin{minipage}[b]{0.45\hsize}
		\centering
		\includegraphics[width=0.8\linewidth]{images/chapter4/position/slam_position_t81_2.png}
		\subcaption{左の真横から見た様子}
		\label{fig:slam_position_t81_2}
	\end{minipage}
	\caption{SLAMを用いる手法における$t=81$の様子}
	\label{fig:slam_position3}
\end{figure}

\begin{figure}
	\begin{minipage}[b]{0.5\hsize}
		\centering
		\includegraphics[width=0.8\linewidth]{images/chapter4/position/slam_position_t81_points.png}
		\subcaption{点群全体の様子}
		\label{fig:slam_position_t81_points}
	\end{minipage}
	\begin{minipage}[b]{0.35\hsize}
		\centering
		\includegraphics[width=0.6\linewidth]{images/chapter4/position/slam_position_t81_points2.png}
		\subcaption{点群の左羽根付近の様子}
		\label{fig:slam_position_t81_points2}
	\end{minipage}
	\caption{SLAMの出力として得られた$t=81$の疎な点群}
	\label{fig:slam_position4}
\end{figure}

\begin{figure}[H]
	\centering
	\begin{minipage}[t]{0.45\linewidth}
		\centering
		\includegraphics[width=0.9\linewidth]{images/chapter4/position/absrel2.png}
		\subcaption{Abs Rel}
		\label{fig:absrel12}
	\end{minipage}
	\begin{minipage}[t]{0.45\linewidth}
		\centering
		\includegraphics[width=0.9\linewidth]{images/chapter4/position/mae2.png}
		\subcaption{MAE}
		\label{fig:mae2}
	\end{minipage}
	\begin{minipage}[t]{0.45\linewidth}
		\centering
		\includegraphics[width=0.9\linewidth]{images/chapter4/position/rmse2.png}
		\subcaption{RMSE}
		\label{fig:rmse2}
	\end{minipage}
	\caption{SLAMを用いる手法の位置座標推定の精度}
	\label{fig:slam_position_result}
\end{figure}

\subsection{動的物体の追跡の精度評価}\label{subsec:tracking_eval}
本節では,オプティカルフローと単眼深度推定を組み合わせた比較手法と,動的物体にSLAMを適用する提案手法の二手法で追跡の精度評価を行う.
どちらも$t=51$の正解点のデータを初期値として与え,$t\geqq51$での各点の三次元座標を推定した.
そして,推定点の三次元座標と正解データの三次元座標の誤差を追跡の精度として評価した.
評価指標は前節と同じAbs Rel, MAE, RMSEの三指標を用いた.

\subsubsection{比較手法の追跡の精度評価}\label{subsubsec:2d_tracking_eval}
比較手法ではオプティカルフロー推定と単眼深度推定を組み合わせることで動的物体の追跡を実現した.
\ref{subsec:opticalflow_eval}節にあるように,$t=51$での正解データの画像座標を初期値として,オプティカルフローの推定結果を逐次的に用いることで画像上で各点を追跡した.
こうして追跡された画像座標をもとに単眼深度推定の結果を用いて三次元再構成した推定点の精度を図\ref{fig:monodepth_tracking_result}に示す.
三指標のグラフの形状はほぼ同じであり,前節の深度推定を用いた深度推定の結果である図\ref{fig:monodepth_position_result}の$t\geqq51$の部分とすべての指標において近い値となった.
これは,オプティカルフロー推定を逐次的に行うことで累積した画像に平行な方向の誤差よりも,単眼深度推定によって生じた深度方向の誤差の方が,追跡精度に相対的に大きな影響を及ぼしたということである.

\begin{figure}[H]
	\centering
	\begin{minipage}[t]{0.45\linewidth}
		\centering
		\includegraphics[width=0.9\linewidth]{images/chapter4/tracking/absrel1.png}
		\subcaption{Abs Rel}
		\label{fig:tracking_absrel1}
	\end{minipage}
	\begin{minipage}[t]{0.45\linewidth}
		\centering
		\includegraphics[width=0.9\linewidth]{images/chapter4/tracking/mae1.png}
		\subcaption{MAE}
		\label{fig:tracking_mae1}
	\end{minipage}
	\begin{minipage}[t]{0.45\linewidth}
		\centering
		\includegraphics[width=0.9\linewidth]{images/chapter4/tracking/rmse1.png}
		\subcaption{RMSE}
		\label{fig:tracking_rmse1}
	\end{minipage}
	\caption{オプティカルフロー推定と単眼深度推定を用いる手法の追跡精度}
	\label{fig:monodepth_tracking_result}
\end{figure}

\subsubsection{提案手法の追跡の精度評価}\label{subsubsec:3d_tracking_eval}
提案手法では,SLAMにより得られた各時刻のカメラの相対自己位置を用いて,初期位置である$t=51$の三次元位置を座標変換していくことで三次元空間上での追跡を可能とした.
本研究では追跡に用いる三次元点の初期位置として以下の二つの条件を用い,それぞれについて評価した.
\begin{itemize}
\item [条件1]$t=51$の正解データの画像座標をSLAMの出力を元に三次元再構成した三次元座標
\item [条件2]$t=51$の正解データの三次元座標
\end{itemize}
条件1は$t=51$の正解データの画像座標をもとに,$t=51$でのSLAMの出力を用いて推定された三次元座標を,各時刻のカメラの相対自己位置を用いて追跡(座標変換)していく.
そのため,$t=51$におけるSLAMの三次元復元の精度および,$t\geqq51$のカメラの相対自己位置の推定精度の両者が影響する.
一方で,条件2での出力は正解データの三次元座標をカメラの相対自己位置を用いて座標変換したものであり,SLAMにより得られたカメラの相対自己位置のみが影響する精度評価となる.

各時刻での両条件の精度評価の結果を図\ref{fig:slam_tracking_result}に示す.
まずは正解データの画像座標を初期値として与えた条件1についての結果を確認する.
すべての時刻を通じてAbs Relは0.0014以下,MAEは1.2 mm以下,RMSEは1.7 mm以下と高い精度で追跡が行えている.
図\ref{fig:monodepth_tracking_result}で示した比較手法の結果と比べると,すべての時刻を通じて90 \%以上の誤差の低減が確認された.
三指標ともに時刻による誤差の変化は小さく,$t=51$での誤差とほぼ同じ値を取った.
これは$t=51$において正解データの画像座標から再構成された三次元点群の誤差が,以降追跡においても強く影響したことが原因と考えられる.
実際に,真横から見た場合に$t=51$の時点で生じていた深度方向の誤差が大きい点は,その後の時刻においても他の点と比べて誤差の大きい点であり続けている(図\ref{fig:slam_tracking_jouken1}).
また,正面から見た場合は推定点と正解点がほとんど重なっていることから,カメラ画面に平行な方向と比べて,カメラ深度方向の誤差がSLAMを用いた提案手法の追跡で支配的であるといえる.

次に正解データの三次元座標を直接初期値として追跡を行った条件2についてである.
すべての時刻を通じてAbs Relは0.0011以下,MAEは1.0 mm以下,RMSEは1.1 mm以下と,こちらも条件1と同様に高い精度で追跡が行えている.
条件1と比べて各時刻のMAEとRMSEの差が小さく,相対自己位置の誤差が原因で,小さな誤差を持つ点が全体的に生じていることがわかる.
三指標ともに,$t=51$時点から$t=77$付近までにかけて上昇し,その後$t=99$にかけて緩やかに誤差が減少している.
$t=51$から$t=77$までというのは入力動画の中で扇風機の羽根が半周ほど回転する時間間隔であり,$t=77$付近は$t=51$との相対自己位置の距離の差が最も大きくなる時刻である.
SLAMを行う際に,以前の時刻に取得された特徴点との対応付けが行われているため,$t=51$の初期位置に近い部分ほど,初期位置の付近で見つかった特徴点との対応付けがしやすく,自己位置の推定精度が上がると考えられる.
このことから,$t=51$時点との相対自己位置の距離の差が最も大きくなる$t=77$付近で誤差が上昇し,その後羽根が初期位置に近づくにつれて誤差が減少したと考えられる.

また,条件1と条件2のグラフを比べると,$t=56, 58, 96$のような点で,両条件で誤差が一時的に上昇しているのがわかる.
条件2で誤差が大きくなるということは,その時刻では相対自己位置推定の誤差が大きくなっているということである.
つまり,条件1では$t=51$時点の初期位置の誤差が支配的ではあるものの,相対自己位置推定の誤差の影響も一定程度受けていることが確認された.

\begin{figure}[h]
	\centering
	\begin{minipage}[t]{0.45\linewidth}
		\centering
		\includegraphics[width=0.9\linewidth]{images/chapter4/tracking/t_51.png}
		\subcaption{$t=51$(初期値)}
		\label{fig:tracking_t51}
	\end{minipage}
	\begin{minipage}[t]{0.45\linewidth}
		\centering
		\includegraphics[width=0.9\linewidth]{images/chapter4/tracking/t_60.png}
		\subcaption{$t=60$}
		\label{fig:tracking_t60}
	\end{minipage}
	\begin{minipage}[t]{0.45\linewidth}
		\centering
		\includegraphics[width=0.9\linewidth]{images/chapter4/tracking/t_70.png}
		\subcaption{$t=70$}
		\label{fig:tracking_t70}
	\end{minipage}
	\begin{minipage}[t]{0.45\linewidth}
		\centering
		\includegraphics[width=0.9\linewidth]{images/chapter4/tracking/t_80.png}
		\subcaption{$t=80$}
		\label{fig:tracking_t80}
	\end{minipage}
	\begin{minipage}[t]{0.45\linewidth}
		\centering
		\includegraphics[width=0.9\linewidth]{images/chapter4/tracking/t_90.png}
		\subcaption{$t=90$}
		\label{fig:tracking_t90}
	\end{minipage}
	\begin{minipage}[t]{0.45\linewidth}
		\centering
		\includegraphics[width=0.9\linewidth]{images/chapter4/tracking/t_99.png}
		\subcaption{$t=99$}
		\label{fig:tracking_t99}
	\end{minipage}
	\caption[条件1の三次元復元結果を二つの視点から見た様子]{条件1の三次元復元結果を正面(左図)と真横(右図)から見た様子.白が正解点,緑が推定点.正面から見た際はずれが少ないのに対し,奥行き方向にはすべての時刻で大きくずれている点が複数ある.}
	\label{fig:slam_tracking_jouken1}
\end{figure}

\begin{figure}[h]
	\centering
	\begin{minipage}[t]{0.6\linewidth}
		\centering
		\includegraphics[width=0.9\linewidth]{images/chapter4/tracking/absrel5.png}
		\subcaption{Abs Rel}
		\label{fig:tracking_absrel2}
	\end{minipage}
	\begin{minipage}[t]{0.6\linewidth}
		\centering
		\includegraphics[width=0.9\linewidth]{images/chapter4/tracking/mae5.png}
		\subcaption{MAE}
		\label{fig:tracking_mae2}
	\end{minipage}
	\begin{minipage}[t]{0.6\linewidth}
		\centering
		\includegraphics[width=0.9\linewidth]{images/chapter4/tracking/rmse5.png}
		\subcaption{RMSE}
		\label{fig:tracking_rmse2}
	\end{minipage}
	\caption{SLAMを用いる手法の追跡精度}
	\label{fig:slam_tracking_result}
\end{figure}
% \clearpage

% %----------第5章----------
% \section{結論}\label{sec:conclusion}
\subsection{本研究の成果}\label{subsec:result}
本研究では動的物体に対してセグメンテーションをしたうえでVisual SLAMを適用することで,動的物体の三次元形状・運動を復元する手法を構築した.
そして,扇風機の羽根が回転する様子を撮影した動画に対して提案手法を適用し,単眼深度推定を用いる手法と比較することで,三次元形状の推定および追跡における有効性を検証した.
\begin{comment}
	\vskip.5\baselineskip
	\scalebox{1.2}{\textbf{動的物体に対するSLAMの適用手法の構築}}
\end{comment}
\subsection*{動的物体に対するSLAMの適用手法の構築}

従来Visual SLAMは撮影された周辺環境のうち静止している部分に適用されていたのに対し,本研究では動的物体に適用することで動的物体の三次元復元を可能とした.
具体的には,セグメンテーションを行い動的物体上の特徴点を抽出した上でVisual SLAMを適用することで,動的物体の三次元形状・運動を復元する手法を構築した.
セグメンテーションには半教師ありビデオオブジェクトセグメンテーションを用いることで,注目したい動的物体を人によって指定することが可能な手法とした.
セグメンテーションにおいて物体検出自体は行わない一方で,未学習の動的物体に対しても適用することができ,提案手法の汎用性は高いといえる.

\subsection*{実際の撮影データを用いた有効性の検証}

扇風機の羽根が回転する様子をハイスピードカメラで撮影した動画に対して提案手法を適用し,その有効性を検証した.
その際,深層学習をベースとした単眼深度推定を用いる手法と精度の比較を行った.
三次元位置座標の推定では,単眼深度推定を用いる方法と比べて90 \%以上誤差が低減し,推定精度の大幅な向上が確認された.
その一方で,SLAMにより生成された点群に外れ値となるような点が含まれていた場合,その点の付近で精度が低下するケースが生じた.
動的物体の追跡についても,オプティカルフロー推定と単眼深度推定を組み合わせた比較手法と比べて90 \%以上誤差が低減し,大幅な精度の向上が確認できた.
その際,動的物体に対するカメラの相対自己位置の変化が大きくなるにつれて相対自己位置推定の精度が下がり,それに伴って動的物体追跡の精度も低下することが確認された.

\subsection{本研究の課題,展望}\label{subsec:obstacles}
以上のような成果を確認した一方で,より汎用的な手法を構築するうえで以下のような課題,発展の可能性が挙げられる.
\subsection*{カメラが動く条件への適用}

本研究で検証に用いたデータは静止したカメラから撮影された動画であった.
しかし,本研究の提案手法はカメラが動いている場合でも適用することが可能である.
具体的には,セグメンテーションによって静止している部分の特徴点を抽出しSLAMを適用するという従来の手法により,絶対座標系におけるカメラの自己位置を得ることができる.
この結果と動的物体に対してSLAMを適用した結果を組み合わせることで,カメラが動いている状態であっても動的物体の位置座標,運動復元が可能となる.
カメラの動きがある場合でも適用することで,カメラが揺れる条件でのモニタリングや自動運転といった分野にも応用が可能になるといえる.
\subsection*{変形を伴う動的物体を扱うための手法の拡張}

本研究の提案手法では,SLAMを用いる際に動的物体は剛体であるという仮定が必要であった.
しかし,構造物のモニタリングでは変形を伴う物体の三次元復元を行いたい例も多い.
注目している動的物体の一部が変形を伴う場合には,SLAMにより得られた相対自己位置をもとに,動的物体のうち変形している部分に関してのみ三次元位置を再計算するといった手法の拡張が必要である.
例えば,本研究で高い推定精度を示したオプティカルフロー推定と相対自己位置を組み合わせるということで変形した部分の三次元座標を推定するといった方策が考えられる.

\subsection*{テクスチュアの少ない動的物体への適用}

本研究の検証に用いた扇風機の羽根にはドットパターンが付与されていた.
そのため,扇風機の羽根の内部でも特徴点が検出しやすく,羽根の表面の三次元点群をSLAMにより比較的密に得ることができた.
しかし,実在する風車等を考えればこのようなドットパターンが付与されていることはなく,SLAMを行う際に精度が急激に低下する可能性が考えられる.
また,動的物体の内部でにテクスチュアがない場合,特徴点は動的物体と静止している背景領域の境界付近で検出されやすくなる.
こうして検出された境界付近の動的物体の特徴点がセグメンテーションによって排除されないよう,動的物体のマスクの領域を拡大するといった方法も考えられる.
\subsection*{動的物体の三次元復元で生じる外れ値の扱い}

動的物体にSLAMを適用することで生成された疎な三次元点群は,大きな外れ値となる点を含んでおり,これらの点が三次元位置推定の精度を大きく低下させていた.
本研究では深度方向に閾値を設けることで大幅に外れている点を除去する操作を行ったが,閾値の内部に収まっていても推定した表面形状から外れている点がいくつか存在した.
深度方向の閾値を指定する方法以外にも,近傍点が少ない点を外れ値として除外するなどの方法をとることで三次元位置推定の精度のさらなる向上が期待できる.
% \clearpage

%----------参考文献----------
\pagestyle{fancy}
\rhead{}%ヘッダ右はなし
\lhead{\rightmark}%ヘッダ左に参考文献と表示
\bibliography{reference}%bibファイルによる管理
\addcontentsline{toc}{section}{参考文献}%目次に参考文献を加える
\clearpage

%----------謝辞----------
\begin{heightchange}
	\newpage
	\thispagestyle{empty}%ページ番号も入らないので注意
	\input acknowledgements
\end{heightchange}

\end{document}