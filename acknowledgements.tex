\section*{謝辞\markboth{謝辞}}
	\addcontentsline{toc}{section}{謝辞}%目次に謝辞を加える
	
本論文を書くにあたって,様々な方にお世話になりました.
	
特に指導教員である布施先生は,何もわからない状態であった私に対して常に道を示し続けてくださりました.
実装のアイディアが浮かばないとき,本研究の学術的な意義を見いだせずにいたときなど,何度も先生に助けられました.
また,布施先生が提示していただいた本研究のテーマは非常に興味深く,どうすればより良いものになるか,どうすれば本研究の意義を見出せるのかを常に考えることができました.
深く御礼申し上げます.

助教である安田先生をはじめとする地域情報研究室の皆さんにも大変お世話になりました.
安田先生は学生ゼミだけでなく日常的に研究のことを気にかけてくださり,ありがとうございました.
河内先輩は直接研究内容についてお話しできる機会は少なかったですが,教えていただいた文献などを勉強することで研究をするために必要な力をつけることができました.
学生ゼミでの木村先輩の的確な質問には,研究のロジックの部分を何度も考えさせられました.
研究内容以外のことで木村先輩とお話したことも,非常にためになることばかりでした.
邱さんや片山先輩は,パソコンの使い方がほとんどわからない状態で入ってきた私に様々なことを教えてくださりました.
特に邱さんがワールドカップ決勝が始まる深夜まで環境構築を手伝ってくださったこと,その次の日にプログラムがうまく回って片山先輩と喜んだことは,とても印象に残っています.
エラー続きであった実装も何とか乗り切ることができたのは,お二人の手厚く丁寧なサポートがあったおかげです.
本当にありがとうございます.
	
また,地域情報研究室のOBである稲福先輩にもとてもお世話になりました.
昨年,ちょうどこの時期に行われた稲福先輩の卒業論文の発表を聞いて地域情報研究室に興味を持ち,卒業論文でも同じようなテーマを選ばせていただきました.
研究を進める上でも,自分が陥っている課題やそれに対する改善策を相談させていただけたことは,とても大きな助けになりました.
同じようなテーマで研究され,悩みを共有することができる方の存在は非常に大きかったです.
稲福先輩とZoomやSlackで話すことがなければ今頃この卒業論文がどうなっていたことか,想像するだけで恐ろしいです.
フランスという遠い地から,忙しい中私を支えてくださったことを深く感謝しています.

本研究の検証で使用するデータをいただいた株式会社 東芝 研究開発センターの皆様にもお礼申し上げます.
いただいたデータは回転羽根をハイスピードカメラで撮影し,物体を三次元復元・追跡するという計測が難しくかつ貴重なデータであり,本研究の検証を行う上で非常に適したデータでした.
様々な撮影条件での計測・解析を行い,必要なデータをそろえていただいたことに深く感謝申し上げます.

そして,大学の間所属していた男子ラクロス部の方々.
私の四年間の大学生活の中心には常にラクロス部がありました.
ラクロス部で得た経験が今の自分を形作っており,そこで得た仲間は私の一生の宝です.
四年間ありがとうございました.

ここには書くことのできなかった方も含め,関わっていただいたすべての方に感謝いたします.

\vspace{\baselineskip}
最後に家族の皆様へ.
いつも,私を支えてくれてありがとうございます.
卒業論文だけでなく部活動,勉強,バイトなどいろいろなことがあった四年間を乗り切れたのは家族の皆様の支えがあってこそです.
いくら感謝しても足りませんが,本当にありがとうございます.
そして,これからもよろしくお願いします.
