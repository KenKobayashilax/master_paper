\begin{abstract}

高度経済成長期に建設された全国の土木構造物の老朽化が進みその維持管理が課題となっている中,人的・財政的な制約の下で,効率的な構造物モニタリングの重要性が高まっている.
数あるモニタリング手法の中でも,単眼カメラを用いた画像センシングは,低コストかつ汎用性が高い方法である.
しかし,単眼カメラによって得られた二次元画像は,奥行き方向の距離情報を含んでいないため,原理的には三次元計測を行うことができない.
構造物モニタリングにおいて重要となる三次元計測のためには,ステレオ撮影や,他のセンサから得られた深度の情報を追加すること等が必要になる.
これらによらず,単眼カメラのみからどれだけ三次元情報を復元できるかの限界を知ることは,構造物モニタリングの可能性を考える上で大きな意味を持つ.
一方で,移動するカメラの連続画像から撮影対象物の三次元形状を逐次的に推定する手法は,Visual Simultaneous Localization and Mapping(以下,Visual SLAM)として提案されてきた.
しかし,Visual SLAMは撮影対象が静止しているという仮定の下で成り立つものであり,動きのある物体(以下,動的物体)には適用できない.
動的物体が含まれる環境において,動的物体と静止物体を分離し,静止物体の三次元計測の精度向上が試みられてきたが,動的物体自体の形状や変位の推定は,依然としてチャレンジングな課題である.
動的物体の複雑な動きを高精度で推定できる動的Visual SLAMの構築が模索されている.

以上の背景の下,本研究では,単眼カメラのみを用いて動的物体の三次元的な形状や変位を推定する手法を構築することを目的とする.
提案手法は,セグメンテーション,オプティカルフロー推定,三次元復元,空間変換から構成される.
具体的には,半教師ありセグメンテーションを用いて動的物体と静止物体を分離し,注目する動的物体のみに対して特徴点抽出・追跡を行い(オプティカルフロー推定),対象物の三次元復元を行う.
この三次元点群をカメラ空間において再構成することにより,動的物体の三次元的な形状や変位の計測が可能になる.
また,単眼カメラ画像から深度情報を,特徴点によらずに深層学習ベースで直接推定する手法も検討し,比較対象とした.

提案手法の有効性の確認のため,静止したハイスピードカメラから回転羽根を撮影した動画で検証を行った.
提案手法により復元された回転羽根の三次元座標を用いて評価を行った結果,RMSEで1.63 mm,相対絶対誤差で0.0011と十分に高い精度を得た.
これは,単眼深度推定を用いた手法と比較して,誤差が90 \%以上低減し,提案手法の優位性が示された.
また,着目する点の追跡精度の検証も行った.
その結果,動画の全時刻を通じてRMSEが1.7 mm以下,相対絶対誤差が0.0014以下と高い精度を得た.
以上から,提案手法の有効性が確認された.

今後は,従来のSLAMにより得られるカメラの自己位置を統合することで,カメラの動きがある,より一般的な撮影条件への応用が期待される.
また,本研究では動的物体が剛体であるという仮定が必要であったが,変形のある動的物体に対しても適用できるような手法の構築が待たれる.
\begin{comment}
近年,高度成長期に建設された全国の土木構造物の老朽化が進みその維持管理が課題となる中で,構造物モニタリングの省力化の重要性が高まっている.
数あるモニタリング手法の中でも,単眼カメラを用いたリモートセンシングは諸コストがかからず汎用性が高い手法である一方,単眼カメラによって得られた二次元の画像は奥行き方向の距離の情報を含んでいない.
そのため,撮影対象を三次元的に扱うには,ステレオ撮影や,他のセンサから得られた深度情報と組み合わせる等の工夫が必要になる.
これらの工夫によらず,単眼カメラのみからどれだけ三次元的な情報を復元できるかの限界を知ることは,構造物のモニタリングの可能性を考える上で大きな意味を持つ.
移動するカメラの連続画像から撮影対象物の三次元形状を逐次的に推定することは,Visual Simultaneous Localization and Mapping(以下,Visual SLAM)というタスクにより達成できる.
しかし,Visual SLAMは撮影された周辺環境が静止しているという仮定の下で成り立つものであり,動きのある物体(以下,動的物体)と静止している部分(以下,静止背景)を分離し動的物体の影響を除くことで,近年精度の向上が図られてきた.
しかし,動的物体を含んだVisual SLAMにおいて,動的物体自体の形状や変位の把握は依然としてチャレンジングな課題であり,動的物体の複雑な動きを高精度で推定できる動的Visual SLAMの手法が模索されている.

そこで本研究では,従来まで画像から動的物体を除き静止背景にのみに適用していたVisual SLAMの手法を,あえて動的物体に対して適用することで,その形状および運動を復元する手法を提案する.
具体的には,セグメンテーション技術を用いて動的物体の領域と静止背景の領域を分離し,注目したい動的物体の領域のみに対してVisual SLAMを行う.
Visual SLAMの出力として推定された,動的物体の三次元形状と動的物体に対するカメラの相対的な自己位置を用いて,動的物体の各時刻における三次元位置の復元が可能になる.%土木構造物のモニタリングにおいて,単眼カメラのみからでも動きのある構造物の三次元形状や各部分の変位を捉えることができれば,そこから画像上のひび割れの三次元位置の把握や,注目したい部分の画像間の対応関係の把握などを行うことができるようになると期待される.

この手法を評価するため,本研究では静止したハイスピードカメラから扇風機の羽根が回転する様子を撮影した動画で検証を行った.
各時刻での画像上の点を既知として与え,復元された羽根の三次元座標を用いて精度評価を行った結果,深層学習をベースとした単眼深度推定を用いる比較手法と比べて誤差が90 \%以上低減し,推定精度の大幅な向上が確認された.
最初の時刻での画像上の点のみを既知として与え,それ以降の時刻の対応する点の三次元位置を推定した追跡精度の評価においても,比較手法と比べて誤差が90 \%以上低減し,提案手法の優位性が示された.

今後は,カメラが移動するような撮影条件において,本手法により得られるカメラから見た動的物体の三次元位置と,従来のSLAMにより得られるカメラの自己位置を統合することで,カメラの揺れがある条件下などのより一般的な撮影条件への応用が期待される.
また,本研究では動的物体が剛体であるという仮定が必要であったが,画像間における画素の移動推定等を用いることで,変形のある物体に対しても適用できるような手法の構築が待たれる.
\end{comment}
\end{abstract}

