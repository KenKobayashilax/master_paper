\section{序論}\label{sec:intro}
\subsection{背景}\label{subsec:background}

インフラ構造物の施工から維持管理の段階では大量の画像が撮影されており,それらを用いて現場を直感的に把握できるような画像の管理方法が模索されている.
構造物の状況を記録・把握するうえで,画像を用いることは低コストかつ汎用性が高い方法である.
特に,近年ドローン等の普及に伴い,より多くの視点から画像が取得可能になった. 
しかし,これらの記録画像の管理は電子小黒板と紐づけた階層的な管理がされている程度[1]である.
そのため,得たい情報に対応する画像の特定や,撮影場所の正確な特定,画像間の相互参照には時間がかかり,画像を用いた直感的な状況の把握は難しい.\par

この点において,三次元空間上で(任意視点から)画像を参照できる画像群管理プラットフォームの作成は,現場の効率的な把握に大きく寄与する.
モデル作成により,大量の画像データが一つのモデルに集約されることに加え,画像と撮影位置の対応が明確になり,容易な位置特定が可能になる.
また,モデルから未撮影視点での画像の取得が可能となることで,再撮影等の手戻りが少なくなるとともに,詳細で直感的な現場の把握・管理が実現できる.
特に,深層学習を用いた画像処理による損傷検出[2]等が可能になった現在,フォトリアルな三次元モデルの利用可能性が高まっている.\par

しかしながら,撮影画像を用いた三次元再構成に関する一般的な手法は,1.複雑な形状を軽量なモデルで再現できない,2.視点によって見え方が異なる部分の復元に失敗するという二つの課題がある.
撮影画像群からの三次元再構成では,Structure from Motion(SfM)とMulti View Stereo(MVS)という二つの手法を組み合わせることが一般的である.
撮影画像群を入力として,SfMで疎な三次元点群とカメラ評定要素を推定し,MVSで密な三次元モデルを作っていく.
しかし,一般的なMVSは,ポリゴンや点群など離散的な空間表現を用いるため,細部の複雑な形状を表現するには高い計算コストとデータ容量を要する.
また,反射や透過といった視線方向によって見え方が異なる部分では,SfMで対応する特徴点同士の対応がつかず,復元が失敗するという課題がある.\par

そこで,複雑な形状や視点による見え方の違いも表現できるような,フォトリアル(photo-realistic)な三次元再構成手法として,3D Gaussian Splatting(3DGS)[4]が近年注目を集めている.
3DGSでは,色を持った三次元の異方性ガウシアンの集合として空間を表現する.
異方性を持ったガウス分布を用いることで,点群やメッシュと比べて少ないパラメタで複雑で連続的な表現を可能にしている.
また,各ガウシアンが視点依存の色情報を持つことができるため,反射や透過といった視線方向依存の違いを表現できる.
連続的表現が可能な陰関数を用いた手法[]と比べても,高速なモデル作成・レンダリングが実現でき,コンピュータービジョンの分野で盛んに研究が行われている.\par

しかし,入力画像にピンボケや手ブレといった不鮮明な画像が含まれている場合の三次元再構成は,3DGSであっても依然としてチャレンジングな課題である.
その原因は,1.ガウシアンが入力画像のノイズに過剰に適合すること,2.入力画像からの特徴点抽出が困難であるためにSfMが失敗し,3DGSの入力となるカメラ姿勢が不正確になることが挙げられる.\par

不鮮明な画像の中でも,ピンボケ画像が含まれていた場合に適用可能な3DGS手法の開発は,未だ発展途上である.
現在提案されているピンボケに対応した3DGS手法[5, 6]では,入力画像を再現するようなピンボケの表現が深層学習ベースで学習されている.
そのため,ガウシアンがピンボケに過剰に適合し,カメラからの深度に応じてピンボケの大きさが変化するという幾何学的条件が満たされず失敗する場合がある.
また,ピンボケに対応した3DGS手法で,カメラ評定要素が不正確な場合に修正するモデルは, 私の知る限りはない.\par

そこで,本研究では, 幾何学的条件に基づくピンボケ表現を導入するとともに,不正確なカメラ評定要素の修正を行える3DGS手法を提案する.
ピンボケ画像を含んだ画像群からであってもロバストにフォトリアルな三次元再構成が可能になることで,撮影された画像情報を余すことなく利用でき,より効率的な現場把握が可能になると期待される.

\begin{figure}
    \centering
    \includegraphics{images/3d_SLAM.png}
    \caption[Visual SLAMによる自己位置と三次元周辺環境地図推定の例]{Visual SLAMによる自己位置(青印)と三次元周辺環境地図推定の例(\cite{mur2017orb}より).}
    \label{fig:visual slam example}
\end{figure}



\subsection{目的}\label{subsec:objective}

本研究では,「ピンボケ画像群からの3D Gaussian Splatting手法の構築」を目的とする.
具体的には,カメラの原理に基づくピンボケ表現の導入.
そして,不正確なカメラ姿勢の修正を伴う最適化である.
%そして,動的構造物の各画素における変位を捉え,三次元空間上で追跡することである.

\subsection{本論文の構成}\label{subsec:construction}

本論文は,本章を含めて全5章で構成される.

本章では,本研究の背景・目的を述べた.

第2章では,既往研究の整理を行う.

第3章では,本研究の提案手法を紹介する.

第4章では,提案手法を適用した結果とその精度について議論する.

第5章では,本論文の結論を示した後,今後の発展可能性を示す.