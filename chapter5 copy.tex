\section{結論}\label{sec:conclusion}
\subsection{本研究の成果}\label{subsec:result}
本研究では動的物体に対してセグメンテーションをしたうえでVisual SLAMを適用することで,動的物体の三次元形状・運動を復元する手法を構築した.
そして,扇風機の羽根が回転する様子を撮影した動画に対して提案手法を適用し,単眼深度推定を用いる手法と比較することで,三次元形状の推定および追跡における有効性を検証した.
\begin{comment}
	\vskip.5\baselineskip
	\scalebox{1.2}{\textbf{動的物体に対するSLAMの適用手法の構築}}
\end{comment}
\subsection*{動的物体に対するSLAMの適用手法の構築}

従来Visual SLAMは撮影された周辺環境のうち静止している部分に適用されていたのに対し,本研究では動的物体に適用することで動的物体の三次元復元を可能とした.
具体的には,セグメンテーションを行い動的物体上の特徴点を抽出した上でVisual SLAMを適用することで,動的物体の三次元形状・運動を復元する手法を構築した.
セグメンテーションには半教師ありビデオオブジェクトセグメンテーションを用いることで,注目したい動的物体を人によって指定することが可能な手法とした.
セグメンテーションにおいて物体検出自体は行わない一方で,未学習の動的物体に対しても適用することができ,提案手法の汎用性は高いといえる.

\subsection*{実際の撮影データを用いた有効性の検証}

扇風機の羽根が回転する様子をハイスピードカメラで撮影した動画に対して提案手法を適用し,その有効性を検証した.
その際,深層学習をベースとした単眼深度推定を用いる手法と精度の比較を行った.
三次元位置座標の推定では,単眼深度推定を用いる方法と比べて90 \%以上誤差が低減し,推定精度の大幅な向上が確認された.
その一方で,SLAMにより生成された点群に外れ値となるような点が含まれていた場合,その点の付近で精度が低下するケースが生じた.
動的物体の追跡についても,オプティカルフロー推定と単眼深度推定を組み合わせた比較手法と比べて90 \%以上誤差が低減し,大幅な精度の向上が確認できた.
その際,動的物体に対するカメラの相対自己位置の変化が大きくなるにつれて相対自己位置推定の精度が下がり,それに伴って動的物体追跡の精度も低下することが確認された.

\subsection{本研究の課題,展望}\label{subsec:obstacles}
以上のような成果を確認した一方で,より汎用的な手法を構築するうえで以下のような課題,発展の可能性が挙げられる.
\subsection*{カメラが動く条件への適用}

本研究で検証に用いたデータは静止したカメラから撮影された動画であった.
しかし,本研究の提案手法はカメラが動いている場合でも適用することが可能である.
具体的には,セグメンテーションによって静止している部分の特徴点を抽出しSLAMを適用するという従来の手法により,絶対座標系におけるカメラの自己位置を得ることができる.
この結果と動的物体に対してSLAMを適用した結果を組み合わせることで,カメラが動いている状態であっても動的物体の位置座標,運動復元が可能となる.
カメラの動きがある場合でも適用することで,カメラが揺れる条件でのモニタリングや自動運転といった分野にも応用が可能になるといえる.
\subsection*{変形を伴う動的物体を扱うための手法の拡張}

本研究の提案手法では,SLAMを用いる際に動的物体は剛体であるという仮定が必要であった.
しかし,構造物のモニタリングでは変形を伴う物体の三次元復元を行いたい例も多い.
注目している動的物体の一部が変形を伴う場合には,SLAMにより得られた相対自己位置をもとに,動的物体のうち変形している部分に関してのみ三次元位置を再計算するといった手法の拡張が必要である.
例えば,本研究で高い推定精度を示したオプティカルフロー推定と相対自己位置を組み合わせるということで変形した部分の三次元座標を推定するといった方策が考えられる.

\subsection*{テクスチュアの少ない動的物体への適用}

本研究の検証に用いた扇風機の羽根にはドットパターンが付与されていた.
そのため,扇風機の羽根の内部でも特徴点が検出しやすく,羽根の表面の三次元点群をSLAMにより比較的密に得ることができた.
しかし,実在する風車等を考えればこのようなドットパターンが付与されていることはなく,SLAMを行う際に精度が急激に低下する可能性が考えられる.
また,動的物体の内部でにテクスチュアがない場合,特徴点は動的物体と静止している背景領域の境界付近で検出されやすくなる.
こうして検出された境界付近の動的物体の特徴点がセグメンテーションによって排除されないよう,動的物体のマスクの領域を拡大するといった方法も考えられる.
\subsection*{動的物体の三次元復元で生じる外れ値の扱い}

動的物体にSLAMを適用することで生成された疎な三次元点群は,大きな外れ値となる点を含んでおり,これらの点が三次元位置推定の精度を大きく低下させていた.
本研究では深度方向に閾値を設けることで大幅に外れている点を除去する操作を行ったが,閾値の内部に収まっていても推定した表面形状から外れている点がいくつか存在した.
深度方向の閾値を指定する方法以外にも,近傍点が少ない点を外れ値として除外するなどの方法をとることで三次元位置推定の精度のさらなる向上が期待できる.